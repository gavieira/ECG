% Created 2021-06-17 Thu 19:03
% Intended LaTeX compiler: pdflatex
\documentclass[bigger]{beamer}
\usepackage[utf8]{inputenc}
\usepackage[T1]{fontenc}
\usepackage{graphicx}
\usepackage{grffile}
\usepackage{longtable}
\usepackage{wrapfig}
\usepackage{rotating}
\usepackage[normalem]{ulem}
\usepackage{amsmath}
\usepackage{textcomp}
\usepackage{amssymb}
\usepackage{capt-of}
\usepackage{hyperref}
\usepackage[backend=bibtex,style=apa,autocite=inline]{biblatex}
\bibliography{../Bibliometry.bib}
\usetheme{default}
\author{Gabriel Alves Vieira}
\date{25-06-2021}
\title{Bibliometria e a avaliação da publicação científica}
\hypersetup{
 pdfauthor={Gabriel Alves Vieira},
 pdftitle={Bibliometria e a avaliação da publicação científica},
 pdfkeywords={},
 pdfsubject={},
 pdfcreator={Emacs 27.2 (Org mode 9.5)}, 
 pdflang={English}}
\begin{document}

\maketitle

\begin{frame}[label={sec:org1918bc8}]{Conceitos}
\begin{itemize}
\item Bibliometria:
\item Cientometria:
\item Informetria:
\end{itemize}
\end{frame}

\begin{frame}[label={sec:orgd7e64c2}]{Bibliometria - Leis}
\begin{itemize}
\item Lei de Lotka
\item Lei de Bradford
\item Lei de Zipf
\end{itemize}
\end{frame}

\begin{frame}[label={sec:orgcbfc820}]{Zonas de Bradford}
\begin{itemize}
\item 3 zonas:
\begin{itemize}
\item Cada uma com 1/3 das publicações relevantes
\item Cada uma com N vezes mais periódicos que a anterior
\item N = Constante de Bradford
\end{itemize}
\end{itemize}
\end{frame}

\begin{frame}[label={sec:orgde49abb}]{Science Citation Index}
\begin{itemize}
\item Index: O que é?
\begin{itemize}
\item Forma de recuperação de informações
\end{itemize}

\item Science Citation Index \parencite{garfield1955}
\begin{itemize}
\item Baseado no livro similar de direito
\begin{itemize}
\item Lei é baseada em precedentes
\end{itemize}
\item Recuperar facilmente quem citou
\item Fator de impacto - Medida de importância historica de um artigo
\end{itemize}
\end{itemize}
\end{frame}

\begin{frame}[label={sec:orgb2fbc6d}]{\cite{mugnaini2019}}
\begin{itemize}
\item Distribuição das zonas de bradford no Brasil
\end{itemize}
\end{frame}

\begin{frame}[label={sec:orge61e507}]{ISI Journal Impact Factor}
\end{frame}



\begin{frame}[label={sec:org22fa978}]{Databases}
\begin{itemize}
\item \alert{WoS}
\begin{itemize}
\item 
\end{itemize}

\item \alert{Scopus}
\begin{itemize}
\item 
\end{itemize}

\item \alert{Google Scholar}
\begin{itemize}
\item 
\end{itemize}

\item \alert{Scielo}
\begin{itemize}
\item 
\end{itemize}

\item \alert{Currículo Lattes}
\begin{itemize}
\item 
\end{itemize}
\end{itemize}
\end{frame}


\begin{frame}[label={sec:orge0617f9}]{Indicadores bibliométricos (métricas)}
\begin{itemize}
\item Extremamente usados em estudos bibliométricos e avaliação da produção cientifica
\item Baseados em citações

\item Citações
\begin{itemize}
\item Obsolescencia
\item Curva assimétrica
\item Tempo de acúmulo
\end{itemize}
\end{itemize}
\end{frame}

\begin{frame}[label={sec:org0d386ee}]{Fator de Impacto (FI)}
\begin{itemize}
\item Criado em \ldots{} e atribuído a períodicos

\item Função:
\begin{itemize}
\item 
\end{itemize}

\item Publicado anualmente
\begin{itemize}
\item Inicialmente: ISI
\item Hoje: Clarivate Analytics
\end{itemize}

\item Calculo:
\begin{itemize}
\item 
\end{itemize}
\end{itemize}
\end{frame}

\begin{frame}[label={sec:org2109b85}]{Problemas/limitações do Fator de impacto}
\begin{itemize}
\item Manipulável
\begin{itemize}
\item Inflações artificiais
\end{itemize}

\item Janela de citação de 2 anos
\begin{itemize}
\item Para algumas áreas, não é o bastante
\end{itemize}

\item Baseado na média
\end{itemize}
\end{frame}


\begin{frame}[label={sec:orgfc84c53}]{Índice h ()}
\begin{itemize}
\item 

\item 
\end{itemize}
\end{frame}



\begin{frame}[label={sec:org0b848fb}]{Métricas vs peer-review}
\begin{itemize}
\item Aumento da produção científica
\begin{itemize}
\item Peer-review mais dispendioso
\end{itemize}

\item Maior incorporação das métricas na avaliação
\begin{itemize}
\item País
\item Instituição
\item Departamento
\item Laboratório
\item Pesquisador
\end{itemize}

\item Contratação, financiamento\ldots{}
\end{itemize}
\end{frame}

\begin{frame}[label={sec:org501f090}]{RAE}
\begin{itemize}
\item Grande evento de peer-review
\item Determinação do financiamento na Inglaterra

\item Extremamente criticado
\begin{itemize}
\item Muito caro
\item Estudos mostram correlação entre índices e resultados do RAE
\end{itemize}

\item Será que o peer-review fora das revistas está morto?
\end{itemize}
\end{frame}

\begin{frame}[label={sec:orgc5a55f5}]{Métricas vs peer-review: quadro comparativo}
\begin{center}
\begin{tabular}{ll}
Peer review & Métricas\\
Qualitativo & Quantitativo\\
\end{tabular}
\end{center}
\end{frame}


\begin{frame}[label={sec:orge3ff37e}]{\cite{butler2007}}
\begin{itemize}
\item Abordagem mista
\item Ambos possuem falhas, e se complementam
\item Os dois possuírem correlação em várias áreas serve como possibilidade de refinamento da avaliação
\begin{itemize}
\item Se eles divergirem, os avaliadores devem discutir mais a fundo
\end{itemize}
\end{itemize}
\end{frame}

\begin{frame}[label={sec:org6d01f49}]{Avaliação no Brasil}
\begin{itemize}
\item Qualis
\end{itemize}
\end{frame}


\begin{frame}[label={sec:org809c8d2}]{Avaliação no Brasil 2}
\end{frame}


\begin{frame}[label={sec:orgaff26df}]{Problemas da avaliação baseada em métricas (e em especial no FI)}
\begin{itemize}
\item Mudança do comportamento dos pesquisadores
\begin{itemize}
\item Minar as idéias originais
\item Maior foco em periódicos internacionais de alto impacto
\end{itemize}
\end{itemize}


\begin{itemize}
\item Impacto na integridade científica
\begin{itemize}
\item Evidência anedótica
\end{itemize}

\item Problemas a nível pessoal
\begin{itemize}
\item Produtivismo e competição exacerbada
\item Depressão, burnout
\end{itemize}
\end{itemize}
\end{frame}


\begin{frame}[label={sec:orgbb71e90}]{\cite{demeis2003}}
\begin{itemize}
\item Pesquisa feita no IBQm
\end{itemize}
\end{frame}


\begin{frame}[label={sec:org6991427}]{\cite{demeis2003} - 2}
\end{frame}

\begin{frame}[label={sec:orgcc023df}]{\cite{demeis2003} - 3}
\end{frame}


\begin{frame}[label={sec:org7850a3b}]{Os manifestos}
\begin{itemize}
\item Várias iniciativas surgiram para lidar com o mal uso das métricas nos sistemas de avaliação da política científica
\begin{itemize}
\item DORA
\item Leiden
\item Tide
\item Hong Kong
\end{itemize}
\end{itemize}
\end{frame}

\begin{frame}[label={sec:org5baed58}]{DORA}
\begin{itemize}
\item Desenvolvido e \ldots{} em 2013
\item Site - xxxx assinaturas
\item Recomendações para pesquisadores, instituições, agências de fomento
\end{itemize}
\end{frame}

\begin{frame}[label={sec:orgf1793f0}]{Principal recomendação}
\begin{itemize}
\item Não usar métricas a nível de periódico (como o fator de impacto) para avaliar pessoas.
\end{itemize}
\end{frame}

\begin{frame}[label={sec:org8ba643c}]{Conquistas DORA \parencite{schmid2017}}
\begin{itemize}
\item 

\item 
\end{itemize}
\end{frame}

\begin{frame}[label={sec:org18f37f9}]{Conclusão e perspectivas}
\begin{itemize}
\item A bibliometria é um campo essencial
\item Bibliometria e peer-review podem coexistir
\item Múltiplos inputs (ex: indices clássicos + altmetricos) tendem a prover análises mais holísticas
\item O \alert{mal uso} de qualquer indicador para avaliação invariavelmente terá impactos danosos ao avaliado
\item Temos responsabilidade e vós nesse assunto
\end{itemize}
\end{frame}


\begin{frame}[fragile,allowframebreaks,label=]{Referências}
\printbibliography
\end{frame}

\begin{frame}[label={sec:org8e87141}]{Who watches the watchmen?}
\end{frame}
\end{document}
