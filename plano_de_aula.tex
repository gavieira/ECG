% Created 2021-06-09 Wed 17:02
% Intended LaTeX compiler: pdflatex
\documentclass[11pt]{article}
\usepackage[utf8]{inputenc}
\usepackage[T1]{fontenc}
\usepackage{graphicx}
\usepackage{grffile}
\usepackage{longtable}
\usepackage{wrapfig}
\usepackage{rotating}
\usepackage[normalem]{ulem}
\usepackage{amsmath}
\usepackage{textcomp}
\usepackage{amssymb}
\usepackage{capt-of}
\usepackage{hyperref}
\author{John Doe}
\date{\today}
\title{Plano De Aula}
\hypersetup{
 pdfauthor={John Doe},
 pdftitle={Plano De Aula},
 pdfkeywords={},
 pdfsubject={},
 pdfcreator={Emacs 27.2 (Org mode 9.5)}, 
 pdflang={English}}
\begin{document}

\maketitle
\tableofcontents


\section{Organização apresentação}
\label{sec:org4ddc968}
\begin{itemize}
\item Mote da apresentação:
\item \textbf{“Fifth, we must be aware that often problems are caused not by the data or metrics themselves, but by their inappropriate use either by academics or by administrators (Bornmann \& Leydes- dorff, 2014; van Raan, A., 2005b). There is often a desire for  “quick and dirty” results and so simple measures such as the h- index or the JIF are used indiscriminately without due attention being paid to their limitations and biases” - \cite{mingers2015}}
\item Se eu pudesse refazer o ECG, iria focar total no Fator de impacto (q é do interesse de todos) e elaboraria a partir daí
\item Não adicionei análise de patentes, resolvi focar nas publicações, que julguei ser mais interessante para o meu público alvo
\end{itemize}

\subsection{Introdução/história da bibliometria}
\label{sec:org61a5294}
\begin{itemize}
\item \cite{mingers2015} -  Scientometrics is the study of the quantitative aspects of the process of science as a communication system. \textbf{It is centrally, but not only, concerned with the analysis of citations in the academic literature.}
\item Primórdios da bibliometria
\item Leis bibliométricas (talvez falar só de Lotka e Bradford)
\begin{itemize}
\item OBS: tanto Lotka quanto bradford demonstram a assimetria da distribuição dos artigos: Se plotarmos um  histograma,  independente se o eixo x é o número de autores ou de revistas, o eixo y (nº de publicações) atinge o pico bem no começo (curva deslocada para a equerda).
\item Lotka
\begin{itemize}
\item Usar para introduzir a natureza assimétrica de distribuição de publicações, enquanto atenta que \textbf{um fenômeno muito similar ocorre na distribuição de citações}
\end{itemize}

\item Bradford
\begin{itemize}
\item Citar que é usada até hj (\cite{mugnaini2019}), aproveitando o gancho para falar das diferenças entre as diferenças áreas, algo que vai ser mto importante ao longo da apresentação
\item Lei da dispersão explica pq os índices têm dificuldade em atingir cobertura completa de assuntos. As 2 zonas externas (e especial- mente a terceira) possuem um número muito grande de periódi- cos. Por isso que thompson2015 diz que essa lei inspirou Garfield a criar o SCI, focando em periódicos mais relevantes (core). \textbf{Daqui, ir direto para Eugene Garfield}
\end{itemize}

\item Zipf
\end{itemize}
\item Eugene Garfield, ISI Journal Impact Factor
\begin{itemize}
\item Falar como a lei de Bradford (e Garfield) foram importantes para auxiliar na decisão quanto à aquisição, descartes, encadernação, de- pósito, utilização de verba, planejamento de siste- ma.”
\end{itemize}
\item Talvez falar do surgimento de grupos de pesquisa dedicados (Leiden) e surgimento de periódicos (scientometrics) e eventos específicos da área\ldots{}
\item Contexto histórico:
\begin{itemize}
\item Bibliometria avança graças ao maior valor dado à informação após segunda guerra mundial
\end{itemize}
\item Avanço/barateamento da informática:
\begin{itemize}
\item Ampliação de bancos de dados bibliométricos
\item Ampliação das possíveis aplicações da bibliometria
\begin{itemize}
\item Mapeamentos gŕaficos e modelagem matemáticas
\end{itemize}
\item Desenvolvimento da área
\item Embrião do que foi chamado de ``bibliometria de escritório''
\begin{itemize}
\item Movimento open source (novas ferramentas) e open access (mais sobre isso em \cite{mugnaini2019}) tbm desempenham funções importantes no desenvolvimento da bibliometria.
\end{itemize}
\end{itemize}
\item Principais Bancos de dados:
\begin{itemize}
\item Wos
\item Scopus
\item Scholar
\item Falar de caracteristicas dos bancos de dados, como cobertura, qualidade da informação, etc\ldots{}
\end{itemize}

\item Talvez falar de: Desenvolvimento no Brasil?
\begin{itemize}
\item IBICT - Primeiro indício de institucionalização
\item Scielo
\item Periodicos CAPES
\item Qualis
\begin{itemize}
\item Atentar para como indicadores das bases internacionais compõe a avaliação da produção científica brasileira.
\item \cite{mugnaini2014}
\begin{itemize}
\item Qualis: Mesmo em ciências sociais, o artigo costuma ter mais peso que os livros. Há tantas outras áreas (ciência dura, em sua maioria), que não propõem critérios para a classificação de livros.
\item “A avaliação da produção brasileira não se baseia nas citações que sua produção recebe, mas sim nas citações recebidas pelos periódi- cos onde os brasileiros publicam, principalmente o Fator de Impacto JCR [3a], mesmo considerando literatura extensa sobre suas limitações (ARCHAMBAULT e LARIVIÈRE, 2009; VANCLAY, 2011). Assim, a pouca inserção da produção científica nacional (LETA, 2011) acarreta  numa avaliação baseada em indicadores de produtividade, que resulta em produtivismo exagerado, impondo a necessidade de estabelecimento de critérios de qualidade.”
\item “Como pode-se perceber todas as áreas de avaliação de Biológicas e Engenharias executam a classificação dos periódicos de sua área sim- plesmente manejando a lista de periódicos e respectivo indicador, tendo que atualizar a lista e os parâmetros de cada estrato, a cada triênio.”
\end{itemize}
\end{itemize}
\end{itemize}
\end{itemize}


\begin{itemize}
\item Plataforma Lattes (lattes é provavelmente a fonte de informação com maior abrangência da produção científica nacional) - Usada em mugnaini 2013

\item Falar dos mais diversos usos da bibliometria
\begin{itemize}
\item Lembrar de falar que a bibliometria pode ser usada para fazer n out- ras coisas, como avaliar quais papers são mais interessantes, traçar a evolução de um tópico por meio dos seus artigos principais, etc\ldots{}
\item Plágio
\item Mapeamentos gŕaficos e modelagem matemáticas
\item Redes de citações
\begin{itemize}
\item Se não há citação, não há relação
\item Pesquisadores da mesma área que não se citam não tem similaridades identificadas
\item Pode ser complementada pela análise de linguagem (e vice-versa)
\end{itemize}
\item Análise de linguagem:
\begin{itemize}
\item Co-ocorrências de palavras
\item Natural Language Processing (subcampo de machine learning) + aumento da disponibilidade de textos integrais
\begin{itemize}
\item Ampliação de possibilidades de estudos na área
\end{itemize}
\end{itemize}
\item Entretanto, existe nenhum desses é tão usado para a avaliação da produção científica quanto a análise de citações
\end{itemize}
\end{itemize}

\subsection{Análise de citação}
\label{sec:org41326e9}
\begin{itemize}
\item Usada como a principal medida de prestígio
\item Discutir \textbf{o que a citação realmente representa}
\begin{itemize}
\item Citação indica o número de outros autores para os quais o artigo foi útil de alguma forma\ldots{}
\item Seja para argumentar a favor ou contra
\end{itemize}
\item Talvez aqui: Falar das especificidades entre diferentes áreas.
\begin{itemize}
\item Social Sciences and Humanities - Citation data often not available. In part, because of books being the standard communication vehicle instead of articles. This limits the use of bibliometrics for Evaluation and Policy. \cite{mingers2015}
\item Falar do envelhecimento (obsolescencia) das diferentes áreas.
\end{itemize}

\item \cite{wallin2005}
\begin{itemize}
\item “If a relationship between citation frequency and research quality does exist, this relationship is not likely to be linear. The relationship be- tween research quality and citation fre- quency probably takes the form of a J-shaped curve, with exceedingly bad research cited more frequently than mediocre research (Bornstein 1991)”
\item “The conclusion must therefore be that there is no unam- biguous relationship between citation parameters and scien- tific importance and/or quality. If we then assume that there must after all be some sort of relationship, an explanation for these clearly conflicting inves- tigations must therefore be that the relationship is so complex that we have difficulty in capturing it with the tools available to us
\item The problem with these corre- lations is that the two parameters (peer review and number of citations) are probably not independent (Opthof 1997).
\item Ponto interessante: se considerarmos à queima roupa que citações são sinonimo de qulaidade, um artigo ter 0 citações significa um artigo sem qualidade e, como boa parte das publicações não são citadas at all, isso significa que teríamos que aceitar que boa parte da ciência produzida é essencialmente lixo.
\end{itemize}
\end{itemize}



\begin{itemize}
\item O que as citações medem, afinal? \cite{pendlebury2009} - cita livro de Moed \cite{moed2006}

\item Citação como medida de qualidade: Implica assumir que TODO MUNDO lê TODA A BIBLIOGRAFIA da sua área e consegue, sem viéses, sele- cionar apenas os verdadeiramente mais relevantes. Ao mesmo tempo, os viéses se diluem se analisarmos muitas pessoas de uma vez.
\end{itemize}


\subsubsection{Indicadores (métricas) e avaliação de produção}
\label{sec:org16fd035}
\begin{itemize}
\item \textbf{Focar bastante no h-index e, principalmente, no fator de impacto}
\item Métricas (pesquisador)
\begin{itemize}
\item h-index
\begin{itemize}
\item Falar sobre o cálculo
\item Popularidade
\item Vantagens e desvantagens, assim como as métricas geradas para lidar com essas desvantagens
\item \cite{durieux2010} - Possivelmente um bom exemplo para a aula: – “ For example, J.E. Hirsh has reported that the top 10 researchers in physics and biology have quite different h-indexes (46).”
\item \cite{mingers2015} :  Thomas Khun e como o h-index não faz jus a ele at al, por ele ter poucas publicações
\item Toda a literatura concorda que o h-index sozinho é mto cru, e que deve ser usado com outros indicadores.
\item Ou seja, os dois mais conhecidos e usados indexes são amplamente considerados insuficientes para a avaliação da produção científica \cite{mingers2015}
\end{itemize}
\item G-index
\item HC-index
\item Individual H-Index
\item E-index
\item M-index
\item Q-index
\item Métricas (journals)
\begin{itemize}
\item Fator de impacto
\begin{itemize}
\item Excelente capítulo de livro: \cite{vanraan2019}
\item Publicado anualmente pelo Journal Citation Reports (JCR)
\item A publicação do Journal Impact Factor tem copyright. Não é qqr um q pode calcular e publicar ele.
\item Calculo, o pq da popularidade
\item Quem calcula?
\item Será que ele deve ter um peso grande na avaliação e definição de políticas públicas em países cuja publicação científica é sub-representada no contexto internacional? Tipo o Brasil (mugnaini2019 tem algo sobre isso?)
\item Como ele pode ser manipulado
\item Vantagens e desvantagens
\begin{itemize}
\item Falar do uso primordial: Auxiliar bibliotecas/instituições que querem selecionar quais periódicos assinar. - Tirado de: \cite{wallin2005}
\item Falar da distribuição das citações (muito skewed), e como a média não é uma boa medida de centralidade (ela é, no mínimo, misleading) nesse caso. Dar um exemplo com a mediana (3 ou 4 salários de uma empresa). Falar como \textbf{de um ponto de vista de estatística descritiva, a média não é uma boa medida sumarizadora para distribuições não normais}. A distribuição das citações é chamada de \textbf{lognormal.}
\begin{itemize}
\item Poucos pesquisadore com a maioria dos artigos (Lotka)
\item Poucas revistas com grande parte dos artigos (Bradford)
\item De forma análoga, poucos artigos com grande numero de citações
\end{itemize}
\item Assumir que fator de impacto significa qualidade de um dado periodico é muito propenso a erro, já que isso implica “assumir perfeita comunicação na comunidade científica internacional” (Velho, 1986).
\item Entretanto, até que ponto nós não olhamos só para o fator de impacto em vez de pensar onde que o nosso paper irá cumprir melhor a função dele de comunicar nossos achados ao público alvo? Sendo que esse publico alvo não é toda a comunidade acadêmica, mas um subset muito restrito do mesmo\ldots{} Será que vale a pena pegar uma revista geral em vez de uma específica por causa de alguns décimos de diferença do fator de impacto.
\item Falar do Garfield, e de como ele msm diz que o JIF é indicativo, não um valor absoluto. Até que ponto a maior parte da pesquisa não ser tão citada significa que ela não é de qualidade? Mais do que isso, então 99\% da ciência não é de qualidade então? Que trabalho horrível estamos fazendo aqui? Se considerarmos a citação como medida de qualidade, estamos lascados. . .
\end{itemize}

\item Falar como as críticas levaram à criação e adoção de indicadores alternativos.
\item Cited Half-life
\begin{itemize}
\item Taxa de declínio da curva de citação
\item Parece com o conceito de meia-vida para isótopos radioativos msm
\end{itemize}
\item CiteScore
\begin{itemize}
\item SCImago journal rank (SJR)
\item Source-Normalised Impact per Paper (SNIP)
\end{itemize}
\item Eigenfactor metrics
\begin{itemize}
\item Eigenfactor Score (ES)
\item Article INfluence Score (AIS)
\end{itemize}
\item Immediacy Index

\item Outras questões associadas à avaliação quantitativa científica que alteram o valor de  englobam (tem uma ótima revisão, só que com muita estatística - \cite{waltman2016} ) - Tentar sumarizar em 1 ou 2 slides\ldots{}
\begin{itemize}
\item Normalização
\item Janela de citação (diferentes áreas)
\item Banco de dados utilizado/indexação de periódicos\ldots{} - \cite{garner2018} possui uma boa tabela que mostra bem a diferença entre os índices calculados com diversas databases
\item Cobertura de bancos de dados: Relembrar da lei de Bradford para explicar pq é tão difícil um banco de dados conter toda a publicação relevante de uma dada área.
\end{itemize}
\end{itemize}
\end{itemize}
\end{itemize}
\end{itemize}
\subsubsection{Altmetrics}
\label{sec:org1defc99}


\subsection{Bibliometria e peer-review}
\label{sec:orga9a2cb9}
\begin{itemize}
\item \textbf{Daria para explicar o pq o peer-review foi ``tomado'' pela bibliometria e hj é globalmente usado como métrica ``objetiva''} e juntar duas partes (talvez antes de entrar nas metricas)
\item \cite{juznic2010}
\begin{itemize}
\item Fala do dual system of grant approval da slovenia - usa tanto bibliome- tria como peer review.
\item “An important reason for introducing the dual system of grant approval in 2008 was to decrease the burden of administration, at least for the majority of researchers who already have a rich bibliographic record to prove their excellence. At least half of the researchers that are selected for phase two can be pre-selected using bibliometric methods. ”
\item Há DIVERSOS índices. Mas ele deve ser visto como uma medida acessória, dada a quantidade de publicações.
\item ``Informed peer review'' é um termo que aparece com frequencia
\item Falar que
\end{itemize}
\end{itemize}


\subsection{Qualis}
\label{sec:orga78d853}
\begin{itemize}
\item \cite{thompson2015}
\begin{itemize}
\item “Of course, all metrics must be used in context. Bibliometric indexes should generally be used in concert with a thoughtful review by senior colleagues.33, 34” OLHAR ESSAS REFERÊNCIAS DPS
\end{itemize}
\item Esse foco no qualis/fator de impacto leva a modificações do comportamento dos cientistas (nós)
\item Falar como o qualis baseado no FI é extremamente circular, sendo um mantenedor do status quo.
\item Falar como a idéia dos comitês do qualis é mto boa, mas nossa cultura de supervalorização de métricas (em especial do fator de impacto) é um problema
\end{itemize}

\subsection{Manifestos/Mudança de cultura}
\label{sec:orgf1a28a6}
\begin{itemize}
\item Usar o editorial que cita os princípios do DORA \cite{cagan2013}
\item Ou mostrar os progressos descritos em \cite{hatch2020}
\end{itemize}

\subsection{Conclusão}
\label{sec:orge0e0cec}
\begin{itemize}
\item Esperança: o sistema de avaliação feito pela CAPES continua mudando. Logo, é interessante que a comunidade cientifica se engaje em discussões sobre o tema e (ao menos tente) mudar sua cultura.
\item Sempre lembrar que os problemas não são causados pelas métricas em si, mas sim pelo seu uso inapropriado .
\begin{itemize}
\item \textbf{“Fifth, we must be aware that often problems are caused not by the data or metrics themselves, but by their inappropriate use either by academics or by administrators (Bornmann \& Leydes- dorff, 2014; van Raan, A., 2005b). There is often a desire for  “quick and dirty” results and so simple measures such as the h- index or the JIF are used indiscriminately without due attention being paid to their limitations and biases” - \cite{mingers2015}}
\item Essa questão pode não fazer parte da minha área, mas me afeta (e acredito que afeta todos aqui) diretamente.
\item Se nós não nos preocuparmos com isso, ngm vai\ldots{}
\item Produtividade científica acaba sendo encarada como um fim em si msma - Isso gera os mais deiversos problemas
\begin{itemize}
\item Crise de reprodutibilidade, burnout, aumento de retratações
\end{itemize}
\end{itemize}
\item Avaliação deve, pelo menos, ter múltiplos inputs.
\begin{itemize}
\item Aumenta os outputs, dificuldade de visualização
\item Multiplas interpretações (conceitos de amplitude e abertura de indicadores, a avaliação cientométrica convencional tende a ser estreita nessas duas dimensões)
\item Mas tbm permite tomar decisões mais ponderadas
\item Usar os indicadores como ``dispositivos discutíveis, que permitam aprendizado'' (Barré, 2010, pg. 227), não para definir de forma final ``quem é melhor'' ou algo que o valha
\end{itemize}
\item Incorporar análise a nível de artigo pode ser uma alternativa?
\end{itemize}


\begin{itemize}
\item Nova idéia - 5 artigos:
\begin{itemize}
\item 1 geral - Mugnaini2013
\item 1 fator de impacto/métricas - Garner2018 - Fala de diversas métricas, databases e altmetrics
\item 1 sobre peer-review vs bibliometrics (buscar ``informed peer review no google dps'') e 1 sobre qualis OU 2 artigos sobre Qualis (1 do mugnaini)
\item 1 sobre manifestos (e aquele sobre mudar a nossa conduta?) - Usar o Dora
\end{itemize}
\end{itemize}

\bibliography{Bibliometry}
\bibliographystyle{apalike}
\end{document}
