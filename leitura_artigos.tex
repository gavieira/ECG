% Created 2021-06-07 Mon 19:49
% Intended LaTeX compiler: pdflatex
\documentclass[11pt]{article}
\usepackage[utf8]{inputenc}
\usepackage[T1]{fontenc}
\usepackage{graphicx}
\usepackage{grffile}
\usepackage{longtable}
\usepackage{wrapfig}
\usepackage{rotating}
\usepackage[normalem]{ulem}
\usepackage{amsmath}
\usepackage{textcomp}
\usepackage{amssymb}
\usepackage{capt-of}
\usepackage{hyperref}
\author{Gabriel Alves Vieira}
\date{\today}
\title{Leitura Artigos}
\hypersetup{
 pdfauthor={Gabriel Alves Vieira},
 pdftitle={Leitura Artigos},
 pdfkeywords={},
 pdfsubject={},
 pdfcreator={Emacs 27.2 (Org mode 9.5)}, 
 pdflang={English}}
\begin{document}

\maketitle
\tableofcontents



\section{Artigo: 40 anos da bibliografia no brasil \cite{mugnaini2013}}
\label{sec:orgdd6fd2a}
\subsection{Introdução}
\label{sec:orgf90fa09}

\begin{itemize}
\item Uma das coisas mais importantes da bibliometria é entender a complexidade do processo de comunicação científica em que estamos inseridos.

\item Pergunta: Qual a diferença de bibliometria e biblioteconomia?

\item Ciência e seu potencial de propulsionar desenvolvimento socio-economico: Leva à indicadores para mensurar investimento e frutos da ciência e tecnologia (C\&T).

\item Isso, por sua vez, impacta o planejamento do investimento em C\&T. A avaliação contínua e intensa da ciência se torna importante na definição de quem ganha financiamento.

\item ``Numa economia baseada em conhecimento, a publicação da pesquisa científica torna-se a moeda principal (BOURDIEU, 1994) trazendo consigo os índices e valores dela decorrentes.''

\item A avaliação científica apresenta então um dilema: Ela deve se valer da praticidade da bibliometria/indices/bases de dados, que podem incitar um alto grau de produtividade, ao mesmo tempo em que não ignora a excelência da pesquisa acadêmica. Ambos podem ser excludentes, dependendo da situação.

\item A bibliometria é uma especialidade de outra área já estabelecida: A Ciência da Informação.
\end{itemize}

\subsection{Eventos importantes do desenvolvimento histórico da Bibliometria}
\label{sec:org97ae970}

\begin{itemize}
\item As constantes restrições oramentárias para obter acesso a periódicos foi um dos fatores que levou ao surgimento de técnicas matemáticas/estatísticas para a avaliação da produção científica.

\item Primórdios da bibliometria: 1917 - Fracis J. Cole e Nellie B. Eales - ``Estatística bibliográfica''

\item 1927 - Alfred J Lotka: Lei matemática que diz que frentes de pesquisa são representadas por poucos autores muito produtivos.

\item 1928 - P. L. K. Gross, E. M. Gross - Análise baseada em citações (Química). Busca levantar artigos mais relevantes.

\item Décadas seguintes - Duas novas leis (muito conhecidas) são publicadas:

\begin{itemize}
\item \textbf{Lei de dispersão das publicações de artigos em periódicos} (Samuel C. Bradford - 1934)

\begin{itemize}
\item Tbm chamada ``Lei de Bradford''. Amplamente usada na bibliometria científica, já q periódicos são o principal veículo de comunicação.
\end{itemize}

\item Lei de George K. Zipf (1949)

\begin{itemize}
\item Baseada em ranking de frequências de palavras. Muito aplicada na linguística.
\end{itemize}
\end{itemize}

\item 1955 - Eugene Garfield

\begin{itemize}
\item Publica artigo sobre um índice de citação (parece importante, provavelmente esbarrarei nisso depois)

\item No mesmo artigo, menciona a idéia de fator de impacto de periódicos. Aparentemente, esse tema já foi tratado por outros autores antes, mas foi popularizado por esse artigo de Garfield.
\end{itemize}

\item 1960 - Inúmeros marcos:

\begin{itemize}
\item Termo 'bibliometria' popularizado por Alan Pritchard (mas criado bem antes, em 1934, por Paul Otlet).

\item Criação do Science Citation Index (SCI), por Eugene Garfield, que também transforma a Lei de Dispersão de Bradford na ``lei de concentração de Garfield'', criando um núcleo de periódicos principais abrangindo a ciência mundial.

\item Garfiled também fundou o \textbf{Institute for Scientific Information (ISI)}. Por mais que o nome atual seja Clarivate Analytics (e antes disso, Thomson Scientific), a sigla “ISI” é a referência mais popular.
\end{itemize}
\end{itemize}


\begin{itemize}
\item Derek J. de Solla Price: Origem de nova especialidade: a \textbf{Cientometria.}

\begin{itemize}
\item 'A Cientometria foi chamada por Price, em 1969, “ciência das ciências”, por estudar o comportamento das ciências, se atendo não apenas às publicações, mas ao sistema de pesquisa como um todo.'

\item Ou seja, a Bibliometria está inserida na Cientometria.

\item A princípio, análises quantitativas da produção científica não atentavam para a política científica, algo que muda na década de 70
\end{itemize}
\end{itemize}
\begin{itemize}
\item Década de 70: Barateamento de bens associados à informática, ampliação de bancos de dados bibliométricos e grande desenvovlimento na área.

\item Década de 80: ampliação das possíveis aplicações da bibliometria. Mapeamentos gráficos e modelagem matemáticas.
\end{itemize}

\subsection{Inserção e desenvolvimento no Brasil}
\label{sec:orgbefdc8f}

\begin{itemize}
\item Artigo pioneiro: Urbizagástegui-Alvarado (1984)

\begin{itemize}
\item Levantamento de trabalhos bibliométricos - 2 leis mais usadas

\begin{itemize}
\item 50\% usam a Lei de Bradford

\item 14\% usam a Lei de Lotka
\end{itemize}

\item MACHADO (2007), por outro lado, encontra a Análise de Citação como técnica de análise prevalente. Possivelmente a maior disponibilidade de computadores e poder de processamento\ldots{}
\end{itemize}

\item Instituto Brasileiro de Informação em CiênciaTecnologia (IBICT) - primeiro indício de institucionalização

\item Desenvolvimento da bibliometria (contexto histórico): resposta à importância que passou a ser creditada à informação, não apenas científica, após a segunda guerra mundial.

\item ``A ciência da informação recorre à disciplinas métricas, como a bibliometria e, mais recentemente, cientometria e infometria.'' (MACHADO, 2007)

\item ``Num estudo mais recente, Meneghini e Packer (2010) analisaram a produção científica não apenas em Bibliometria, mas incluíram áreas correlatas como a Cientometria, Informetria, Avaliação de Produção Científica, entre outras, e ainda fizeram uso de fontes de informação com maior abrangência da produção científica nacional – o Google Acadêmico e a Plataforma Lattes – não restringindo a publicações da área de Ciência da Informação.''

\begin{itemize}
\item \textbf{Ou seja, Bibliometria e Avaliação da Produção científica podem ser consideradas coisas distintas?}
\end{itemize}

\item Dicas de periódicos (onde os brasileiros publicam mais):

\begin{itemize}
\item Nacional: Ciência da Informação

\item Internac: Scientometrics
\end{itemize}

\item Bibliometria é parte da ciência da informação, mas é útil à comunidade científica como um todo, permitindo compreender e criticar a política científica nacional e seus métodos de avaliação.
\end{itemize}

\subsection{Fontes de informação e indicadores bibliométricos para subsídio à política científica brasileira}
\label{sec:orge6cd250}

\begin{itemize}
\item Brasil da década de 70: Concepção e desenvolvimento de um sistema de desenvolvimento científico e tecnológico. O que gera, por sua vez, uma nova classificação de periódicos (que na verdade foi um fênomeno internacional)

\item No Brasil, surge o sistema de avaliação de periódicos QUALIS, extensamente abordado no artigo\ldots{}

\item Outras iniciativas citadas:

\begin{itemize}
\item Web of Science - Interface de acesso ao Science Citation Index (SCI)

\begin{itemize}
\item O SCI, por sua vez, apresenta 3 versões: (i) Science, (ii) Social Sciences e Arts \& Humanities e (iii) Journal Citation Reports (JCR)

\item WoS correu atrás de periódicos regionais. Ex:

\begin{itemize}
\item `` O total de periódicos brasileiros indexados na WoS, que em 2005 era 27, alcança um total de 132 em 2010 (TESTA, 2011).''
\end{itemize}
\end{itemize}

\item Portal de Periódicos da CAPES (Periodicos nacionais e internacionais)

\item SciELO (Scientific Electronic Library Online) - Foco em periódicos nacionais

\item JCR - O que é?

\item Google Scholar

\begin{itemize}
\item Gratuito (como tudo da google)

\item Alta capacidade de recuperação de artigos não encontrados nos índices tradicionais
\end{itemize}

\item Scopus

\begin{itemize}
\item Da editora comercial Elsevier (assim como o Mendeley)

\item Cobertura abrangente: periódicos nacionais e regionais.

\item Indexa grande número de periódicos nacionais (especialmente após parceria com SCImago)
\end{itemize}
\end{itemize}

\item Indicadores dessas bases passaram a compor a avaliação da produção científica Brasileira. O Qualis passou por mudanças, por exemplo. É importante se manter atento a essas mudanças, que nos afetam diretamente.

\item Índice h

\begin{itemize}
\item Criado por Jorge E. Hirsch para comparação entre pesquisadores.

\item Quase automaticamente adaptado para analisar periódicos

\item Disponibilizado por Web of Science e Scopus.

\item Assim como o fator de impacto, sua simplicidade metodológica fez com que esse índice extrapolasse a bolha dos especialistas em cientometria.

\item Tanto é que vários outros índices criados pela \textbf{International Society for Informetrics and Scientometrics (ISSI)} não são conhecidos pela comunidade científica.
\end{itemize}

\item Muganini e Sales (2011):

\begin{itemize}
\item ìndices de citação são os principais indicadores usados na avaliação da produção científica nacional.

\item Fator de impacto das revistas também é mto considerado, apesar de estar sempre sujeitos à inúmeras críticas

\item Índice H não é tão usado, muito embora componha critérios de algumas áreas, como no caso das ciencias da saude.
\end{itemize}
\end{itemize}

\subsection{Críticas à consagração de um indicador: alternativas e o ferramental metodológico disponível}
\label{sec:org8223829}

\begin{itemize}
\item JCR - Journal Citation Index. Periódico anual que dá informações sobre os mais diversos periódicos das ciencias naturais e sociais, assim como provê os \textbf{fatores de impacto} dessas revistas.

\item A publicação do JCR, desde 1975, pelo ISI, reforça a proeminencia das revistas/artigos/ciencia \emph{mainstream}. Isso, em grande parte, pelo fator de impacto, o qual vem sendo cada vez mais criticado nas mais diversas publicações.

\item 5 críticas predominantes:

\begin{enumerate}
\item Originalmente desenvolvido para desenvolvimento de coleções, não para avaliação;

\item Incomparabilidade, dadas as especificidades entre áreas;

\item Assimetria entre elementos contados no numerador e denominador;

\item Janela de citação de dois anos;

\item Estabelecimento da língua inglesa e centralização americana.
\end{enumerate}

\item O terceiro fator pode ser manipulado por revistas para aumentar seu fator de impacto.

\item Revistas geralmente rejeitam publicações que potencialmente não serão tão citadas pq poucas citações levam a quedas no fator de impacto

\item Há casos mais extremos, como revistas especializadas que passaram a não publicar nenhum artigo de caso clínico, já que esses geralmente não são tão citado. Uma decisão completamente editorial.

\item O corpo editorial da revista usou isso para tentar aumentar de forma forçosa o fator de impacto. É basicamente um jogo editorial para aumentar o numerador e diminuir o denominador do fator de impacto.

\item O denominador restringe os tipos de artigos considerados, mas o numerador não. Logo, outra estratégia para ``inflar'' o fator de impacto é a publicação de bons editoriais, alatamente citáveis.

\item Glänzel e Moed (2002):

\begin{itemize}
\item Autores falam como o uso do fator de impacto está associado à fatores como ``facilidade de compreensão'' (afinal, é apenas a média de citações recebidas pelos artigos do periódico), ``robustez'' e ``rápida disponibilidade''.

\item Mas tbm falam de limitações, como ``a falta de normalização das práticas de referência; a não discriminação de artigos de revisão, que são muito mais citados; a incapacidade de uma única medida de aferir padrões de citação de periódicos; e o problema da frequente utilização do FI isolado do seu contexto (MOED, 2005).''
\end{itemize}
\end{itemize}


\begin{itemize}
\item Também há várias ressalvas sobre o quão estatísticamente válido é usar apenas a média de citações como indicador do impacto de uma revista.

\item Quando o FI é usado, comumente não são usados testes estatísticos para validá-lo (Frank - 2003)

\item Seria a média uma medida adequada, já que há uma grande discrepancia no numero de citações de diferentes artigos? (Colquhoun 2003)

\item Sem falar que algumas áreas simplesmente possuem menos pessoas trabalhando nelas. Logo, o número de citações (e impacto das revistas/artigos) acaba não refletindo a qualidade da pesquisa.

\item Assumir que fator de impacto significa qualidade de um dado periodico é muito propenso a erro, já que isso implica ``assumir perfeita comunicação na comunidade científica internacional'' (Velho, 1986). Assumir que todos os pesquisadores de todas as áreas possuem igual probabilidade de citar o seu artigo, independente da área de atuação dos mesmos.

\item ``Saha, Saint e Christakis (2003) mencionam que o FI reflete a reputação de um periódico, mais do que sua qualidade.'' Eu adicionaria que reflete a popularidade dos periódicos tbm\ldots{}

\item De forma geral, indicadores bibliometricos podem medir impacto, mas não qualidade. E os índices, dentre eles o fator de impacto de periódicos, podem sim ser úteis quando usados cuidadosamente.

\item Todas essas críticas ao FI vem levando à criação e adoção de indicadores alternativos.

\item Boa parte do trabalho dos bibliometristas é converter a informação bibliogŕafica em bibliometrica. ``A primeira, fiel ao manuscrito do autor, com seus códigos, abreviações e erros; e a segunda, exige cuidadosa padronização para garantia da qualidade da análise quantitativa''

\item Com o avanço e barateamento de ferramentas/computadores, vivemos em uma era da ``bibliometria de escritório'', impossível há alguns anos. Uma era onde dá pra fazer coisas interessantes na área com um computador na mão e uma idéiana cabeça.

\item O movimento Open Source e Open Access também desempenham funções importantes nisso.
\end{itemize}

\subsection{Considerações finais}
\label{sec:org63d7e4b}

\begin{itemize}
\item Países cuja publicação científica é sub-representada no contexto internacional devem dar um grande peso a métricas como o Fator de Impacto na definição de políticas públicas de financiamento científico?

\item É mto comum nas agências de fomento a premiação pela publicação em periódicos de ``alto impacto''

\item Apesar disso, o sistema de avaliação feito pela Capes continua mudando e se adequando à novos contextos. Assim sendo, é no mínimo desejável que a comuniade científica se envolva em discussões sobre o tema.

\item O autor sugere que uma possivel causa da diminuição do uso da lei de bradford é o uso das bases de dados, que estabelecem as revistas importantes de uma dada área. Mas será que as bases de dados estão definindo bem as coleções de periódicos especializados? Será que não estamos ``confortaveis'' com o que nos é servido?

\item Mais do que isso, será que os periódicos são uma boa unidade de avaliação? Ou deveríamos passar a considerar a análise a nível de artigo como uma alternativa?
\end{itemize}

\section{Livro: \cite{mugnaini2017a}}
\label{sec:orga18be2b}

\begin{itemize}
\item Pular artigo “Avaliação Institucional na USP”.
\end{itemize}

\subsection{Discussões gerais sobre as características mais relevantes de infraestruturas de pesquisa para a cientometria}
\label{sec:org35c85a4}

\begin{itemize}
\item Bibliometria/cientometria envolvem custos substanciais. Ter uma infraestrutura adequada à pesquisa bibliométrica ``profissional'' tem um grande custo envolvido.

\item Pesquisar diferença entre cientometria, bibliometria, tecnometria (associado à patentes) e altmetria (métrica baseada no uso  de fontes de dados alternativos, comumente associado à recepção da ciência em outras fontes, como redes sociais), webometria\ldots{}
\end{itemize}

\subsection{Rumo a indicadores para ‘abertura’ de políticas de ciência e tecnologia1}
\label{sec:orgcfe4140}

\begin{itemize}
\item Fala na introdução como o uso simplista de metricas vem sendo criticado, e como está havendo um empenho para melhorar a robustez das métricas, com incorporação dos mais diversos dados, como os altmétricos ou a adição de periódicos nacionais/regionais. Em suma, devem ter mais inputs.

\item ``Com isso, as comunidades de indicadores e de políticas da C\&T voltaram a acreditar que a cientometria deve contar com múltiplas fontes de dados que podem proporcionar ‘indicadores parciais convergentes’ (MARTIN; IRVINE, 1983).''

\item Aumentar os inputs tbm aumenta os outputs, o que adiciona possiblidade de interpretações contrastantes e abre margem para viéses (mas ainda assim menos do que quando não há discussão), mas tbm permite tomar decisões mais ponderadas.

\item Cita artigo que \textbf{não é de review} (RAFOLS et al. 2012) para ilustrar o aumento dos inputs e seu impacto nos outputs.

\item Indicadores como ``dispositivos discutíveis, que permitam aprendizado'' (Barré, 2010, pg.227)

\item Duas dimensões:

\begin{itemize}
\item Amplitude: Associada ao número de inputs

\item Abertura: O quanto que os outputs permitem interpretações plurais e opções de políticas contrastantes a serem debatidas. Quanto maior a abertura, menor a tendência de busca por uma única melhor explicação, método ou resposta. É mais holístico.
\end{itemize}

\item A avaliação cientpométrica convencional tende a ser bastante estreita nas duas dimensões.

\item Mesmo quando a análise é ampla, ela perde abertura ao sumarizar os inputs em uma única medida. Essa transformação limita a discussão sobre desempenho e define de maneira inequívoca qual universidade/pesquisador é ``melhor''. Para além disso, essa redução tbm é mais sujeita a viéses.

\item O contrário pode ser verdade: a análise pode não ser ampla, mas conceitualizar/operacionalizar seus outputs de forma a gerar suposições/resultados constrastantes.

\item Apesar de haver desafios associados à maior amplitude ae abertura, como a questão da visualização dos dados (é difícil fazer uma redução de dimensão quando os resultados vêm de fontes de dados muito distintas, muito embora seja possível operacionalizar eles de forma distinta), é importante evitar que ‘Medidas estatísticas tendem a substituir o debate político pelo conhecimento técnico’ (MERRY, 2011). Sob essas circunstâncias, torna-se imperativo que hajam debates mais abertos envolvendo as escolhas normativas cruciais subjacentes aos indicadores (BARRÉ, 2010). Com isso, teremos avaliações mais rigorosas e confiáveis.
\end{itemize}

\subsection{A pesquisa bibliométrica na era do big data: Desafios e oportunidades}
\label{sec:org6a6cf9b}

\begin{itemize}
\item Nem sempre mais é melhor. Dados podem ter diferentes qualidades.

\item Assim como em qqr análise estatística, as conclusoes tiradas da análise se aplicam somente à amostra utilizada. Características das amostras mudam dependendo do subset utilizado (parametros de est. descritiva e outros)

\item Um grande desafio da bigdata bibliometrica é a aplicação de metodologias para limpeza/análise/visualização de dados

\item Visualização dos dados também é complexo. Muitas dimensões, que geralmente devem ser reduzidas ao msm tempo em que mantêm as relações observadas. Várias técnicas são usadas para simplificar a representação dos dados, como \textbf{análise de agrupamento} e \textbf{análise fatorial para dados quantitativos}, assim como \textbf{métodos baseados em linguagem}.

\item A análise dos dados tbm é um ponto complicado, já que a enorme diversidade de análises/metodologias disponíveis permitem que os pesquisadores cheguem a diversos resultados com os mesmos dados. Logo, a adequação da escolha metodológica deve ser justicada caso a caso.

\item Mas as oportunidades são enormes também. Há mtos dados e porgramas disponíveis para trabalhar com esse crescente contingente de informação bibliográfica.

\item Redes de citações: Extremamente usada, mas tem uma desvantagem: se não há citação, não há relação. Assim sendo, pesquisadores que não se conhecem/citam, mas estão na mesma área ou compartilham interesses não têm suas similaridades identificadas por esse tipo de análise.

\item Análise de linguagem: primeiramente, usava co-ocorrência de palavras em títulos ou palavras-chave para estabelecer relações. Hj, com Natural Language Processing e afins, essas análises passam a ser passíveis de serem utilizadas em textos integrais. Também há outras abordagens, como \textbf{comparações baseadas em texto utilizando modelagem de tópicos (WALLACH, 2006)}. Métodos baseados em texto e em citações podem se complementar.
\end{itemize}

\subsection{Avaliação Institucional na USP}
\label{sec:org66ff0dd}

\begin{itemize}
\item Vários pormenores sobre os tipos de avaliação institucional e como esse processo se dá na USP. Não é mto interessante.
\end{itemize}


\subsection{Políticas Públicas em Ciência e Tecnologia no Brasil: desafios e propostas para utilização de indicadores na avaliação}
\label{sec:org4e5cdc8}

\begin{itemize}
\item Fala sobre o sistema de avaliação da pós-graduação brasileira (realizado pela CAPES) em detalhe.

\item Parece um bom artigo para ser adicionado aos principais\ldots{}

\item Artigo recomendado: ``Dez coisas que você deveria saber sobre o Qualis''
\end{itemize}

\section{Artigo \cite{roldan-valadez2019}}
\label{sec:orgd9b67e1}

\begin{itemize}
\item Paper fala sobre o uso de medidas bibliométricas para escolher a revista mais adequada para o seu paper baseado em impacto e prestígio.
\item ``Since there is a journal performance market and an article performance market, each one with its patterns, an integrative use of these metrics, rather than just the impact factor alone, might represent the fairest and most legitimate approach to assess the influence and importance of an acceptable research issue, and not only a sound journal in their respective disciplines.''
\item Autor fala que, apesar das críticas ao fator de impacto das revistas por vários acadêmicos, é fato que os autores geralmente dão grande importância ao fator de impacto em suas decisões sobre onde submeter os manuscritos.
\item Daí, começa a falar sobre as mais diversas métricas. Histórico, como elas são calculadas, quem está interessado na mesma, seus pontos fortes e críticas.
\end{itemize}

\subsection{Metricas (journals)}
\label{sec:org0a22cfe}
\begin{itemize}
\item IF
\begin{itemize}
\item Publicado anualmente pelo Journal Citation Reports (JCR)
\item Descrição bem detalhada sobre o indicador. Ótima tabela que fala das condições que impactam o IF. Fala sobre a questão das cartas e editoriais e como eles podem ser contados no numerador se citados, mas não no denominador.
\item Inclusive, faz sentido esses itens não serem contados, pois normalmente não são citados mesmo, e adicioná-los diminuiria artificialmente o fator de impacto. Entretanto, do jeito que está, as revistas com mais prestígio tendem a ter seu IF inflado (todo mundo envia cartas e geralmente tem bons editoriais). Tbm abre margem para inchar esse valor ao aumentar o numero de editoriais e priorizar artigos de revisão. Não seria mais interessante considerar citações desse tipo de documento separadamente, em outra métrica, que seja?
\item Também fala de como, para o IF, a distribuição das citações é não-paramétrica. E na verdade, menos de 20\% dos artigos concentram mais de 50\% das citações.
\end{itemize}
\item Cited Half-Life
\begin{itemize}
\item Medida da taxa de declínio da curva de citação
\end{itemize}
\item CiteScore (Uma nova forma de se avaliar o impacto das revistas - Está ganhando mta projeção. Talvez a evolução do IF?)
\begin{itemize}
\item Incorpora SCImago journal rank e Source-Normalised IMpact per paper
\end{itemize}
\item SCImago journal rank (SJR)
\begin{itemize}
\item Ao contrário do IF, que não dá peso para as citações, ele dá um peso maior para citações dos journals com maior SJR.
\end{itemize}
\item Source-Normalised Impact per paper (SNIP)
\begin{itemize}
\item Calcula impacto por citação ao mesmo tempo que considera o total de citações de uma área.
\item Janela de publicação maior (3 anos)
\item Permite comparação entre áreas diferentes
\end{itemize}
\item Eigenfactor metrics
\begin{itemize}
\item Consiste no Eigenfactor score e Article Influence. Disponiveis para o JCR depois de 2007
\item Tem uma ótima tabela comparando os dois tbm
\end{itemize}
\item Eigenfactor score (ES)
\begin{itemize}
\item Num de vezes artigos de uma dada revista q foram publicados nos ultimos 5 anos foram citados nesse ano (JCR - IF year).
\item Citações têm peso diferente, dependendo do journal. Journal self-citation removidas (só são consideradas citações de uma revista para outra).
\item Algoritmo complexo, similar a Google Page Rank
\item Calcula disseminção do artigo
\item Não tem denominador. Logo, é sensível à quantidade de itens citáveis. Em outras palavras, revistas com mais artigos tendem a ter ES maior.
\end{itemize}
\item Article INfluence Score (AIS)
\begin{itemize}
\item Baseado no ES
\item Determina a influencia dos artigos de um jornal após os 5 anos da data de publicação dos mesmos.
\item Ao contrário do ES, ele possui numerador e denominador
\item Cálculo: ES dividido pelo num de artigos no jornal, normalizado como fração de todos os artigos
\end{itemize}
\item Immediacy Index
\begin{itemize}
\item Mede o quanto que artigos recentes de um journal são citados. Ou seja, o quão rápido esses papers desses journals estão sendo adotados na literatura.
\end{itemize}
\end{itemize}

\subsection{Métricas (Pesquisador)}
\label{sec:org5db5d27}

\begin{itemize}
\item H-index
\begin{itemize}
\item Combina produtividade e impacto
\item Criado para avaliar autores, mas pode ser usado para qr conjunto de documentos (ex: publicações de um departamento)
\item Os outros indices tentam abordar problemas/limitações específicas do H-index
\end{itemize}
\item G-index
\item HC-index
\item Individual H-index
\item E-index
\item M-index
\item Q-index
\end{itemize}

\subsection{Altmetrics}
\label{sec:orgbe40d20}
\begin{itemize}
\item ``Altmetrics covers not just citation counts but also various other aspects of the impact of an article such as how many data and knowledge bases refer to it, article views, full-text down- loads, Facebook likes, or mentions in social media and news media [78].''
\end{itemize}

\subsection{Proposed method to use bibliometrics}
\label{sec:orgaa955a1}

\begin{itemize}
\item Os autores tbm sugerem um pipeline/metodologia para planejar em qual revista sumeter o trabalho, acompanhar a importancia do paper e ter uma perspectiva sobre a performance anos depois.
\item Essa é uma abordagem mais integrativa, que não se baseia apenas no IF.
\item Talvez ler para entender isso melhor depois\ldots{}
\end{itemize}


\section{Artigo: \cite{wang2019}}
\label{sec:org74cfa22}

\begin{itemize}
\item Bibliometria só diz respeito à avaliação do impacto de artigos dentro da academia, mas não versa sobre a influencia de pesquisadores fora da academia, e o nome dos autores (para além de outros fatores) estão associados ao quanto os papers são citados. A altmetria tbm visa uma medição do impacto da pesquisa em uma esfera mais social. Um dos maiores beneficios da altmetric é seu potencial de mensurar o impacto mais amplo da pesquisa, aquele que vai para além do meio científico.
\item Janela de 3 anos: aparenetemente, bom período para todas as áreas:
\begin{itemize}
\item ``And according to Glänzel (2008), the use of a 3-year citation window is “a good compromise between the fast reception of life science and technology literature and that of the slowly ageing theoretical and mathematical subjects”'' - artigo dos sete mitos
\end{itemize}
\item Usa uma combinação de preditores obtidos a partir de índices bibliométricos e altmétricos nos primeiros dois anos após a publicação dos artigos para predizer quais artigos se tornarão altamente citados. Consegue fazer isso com um bom poder de predição com as abordagens de machine learning utilizadas. Chega à conclusão que os melhores preditores englobam tanto ínidices altmetricos quanto bibliométricos (muito embora os bibliométricos sejam a maioria, usar ambos parece aumentar o poder de previsão).
\end{itemize}


\section{Artigo: \cite{eysenbach2006}}
\label{sec:org8ca404f}

\begin{itemize}
\item Duas opções para tornar o paper público: Open Acess ou Self-Archive.
\item Mas será que o fato do paper estar público tem impacto no quanto ele é citado?
\item OBS: Papers com mais autores: tendem a ser mais citados, isso pode vir de mais autocitações e/ou pelo paper ter de fato maior qualidade (mais pessoas trabalharam nele, afinal)
\item Trabalho mostra que o paper estar em open access aumenta a immediacy - o quanto que ele é citado no começo da sua vida. Mas também parece aumentar a quantidade de citações de forma geral.
\end{itemize}


\section{Artigo (review): \cite{pendlebury2009}}
\label{sec:org63f95b4}

\begin{itemize}
\item Dá um resumo dos marcos históricos da bibliometria. Dos poloneses, passando por Garfield até o estabelecimento de centros de pesquisa na área.
\item Esse autor diz que bibliometria e cientometria são nomes usados para a mesma coisa.
\item Sugere o livro ``Citation Analysis in Research Evaluation'', por Henk F. Moed, como um bom review do campo da bibliometria até o ano de 2005.
\item JCR não tem apenas o IF das revistas. Tem tbm várias outras informações/métricas, como o immediacy index e meia-vida de citação.
\item Logo, usar o fator de impacto sozinho para definir onde publicar ou o valor de um artigo não é justificado, já que mesmo a revista que gera o fator de impacto gera tantas outras métricas que poderiam ser levadas em consideração. (E diga-se de passagem, mesmo o Garfield sempre disse que o fator de impacto é só mais uma métrica, e que não deveria (mas possivelmente seria) ser usada para classificar artigos)
\item Críticas sobre o Fator de Impacto com relação a como esse é usado (erroneamente) para dar juízo de valor sobre artigos individuais devem ser direcionadas às pessoas que fazem esse mal uso da métrica, não à métrica em si\ldots{}
\item ``H-index, v-index, g-index, y-index, Eigenfactor, audi- ence-factor: What is the non-bibliometrican to think of this mélange of measures? It is important to recognize that different measures attempt to answer different questions and that each will emphasize or highlight cer- tain aspects and nuances of a phenomenon. This is not to deny that some measures may be better, in general terms, than others. There is certainly room for advance- ment in terms of new and better measures. But it is also necessary to point out that there is a fallacy in demanding a single-number metric or just one approach to analysis. ''
\item ``O q as citações medem, afinal?'' - Pergunta é abordada em um livro \cite{moed2006}
\item Citações representam noções de uso, recepção, utilidade, influencia, e o nebuloso termo ``impacto''. Citações, entretanto, \textbf{não representam medidas de qualidade}. Para averiguar a qualidade de um trabalho, \textbf{é estrtitamente necessario o julgamento humano}, por mais que ele possa ter seus viéses e problemas.
\end{itemize}

\section{Artigo (review): \cite{waltman2016}}
\label{sec:org13575d9}
\begin{itemize}
\item Fala bastante sobre databases bibliograficas (WoS, Scopus e Google Scholar)
\item Depois, fala sobre indicadores de impacto baseados em citação
\item Paper enorme (não vou passar pra banca), mas explica tudo nos mínimos detalhes. Bom ler para preparar a apresentação
\item Fala muito sobre janela de citação, auto-citação, normalização\ldots{} Muito bom
\end{itemize}

\section{Artigo (review): \cite{tahamtan2016}}
\label{sec:org010de23}
\begin{itemize}
\item Autor fala muito sobre a importância das citações para mensurar  a qualidade de um artigo, sendo que a citação não mede isso\ldots{} Mas dá uma cutucada nessa visão simplista dps. Dps ele até comenta que diferentes pesquisadores terão diferentes opiniões sobre o q exatamente é qualidade.
\item Basicamente, a qualidade da análise estatistica do paper não impacta o numero de citações
\item Paper com equações diferenciais e mtas notas de rodapé tendem a ter menos citações
\end{itemize}

\section{Artigo: \cite{glanzel2008}}
\label{sec:orgb9cb586}
\begin{itemize}
\item Fala sobre 7 mitos da bibliometria
\begin{enumerate}
\item Myth of delayed recognition
\begin{itemize}
\item Na verdade, são raros os artigos que são mais citados depois
\item E por mais que o envelhecimento da informação científica varie de área para área, o autor defende que 3 anos é um bom compromise
\end{itemize}
\item Myth of self-citation
\begin{itemize}
\item Coloca a autocitação como algo comum da ciência e fala dos argumentos para considerá-la. Ao mesmo tempo, fala de como ela serve para análises específicas
\end{itemize}
\item Colaboração como chave para sucesso
\begin{itemize}
\item Por mais que aceite que de fato a colaboração em média aumente a taxa de citações, questiona a idéia de que isso é o bastante para ganhar financiamento e afins, discorrendo sobre os efeitos nocivos desse mito.
\end{itemize}
\item Citações como medida de qualidade
\begin{itemize}
\item Joga a idéia do artigo/citações como a moeda corrente da ciência
\item Defende que citações e fator de impacto está mais associado à recepção do trabalho do que à sua qualidade, mesmo que haja estudos estatísticos que mostrem correlação entre as duas coisas. (Mto embora correlação não implique causalidade)
\item `` Even where the impact factors are not used as immediate evaluation tools, these journal citation measures often serve as decision criterion and reference standard in the choice of journals for paper submission. Reaching the targeted readership has become a secondary aspect in individual publication strategies''
\end{itemize}
\item Reviews inflam o impacto
\begin{itemize}
\item É verdade que a média de citações de artigos de revisão é maior q a de artigos originais
\item Entretanto, a distribuição de citações de revisões tbm é assimétrica, logo não é uma garantia de alto impacto
\item Para além disso, preparar artigos de revisão é trabalho duro, que requer experiencia na área
\end{itemize}
\item Uma vez altamente citado, sempre altamente citado
\begin{itemize}
\item Fala de exemplos em q msm artigos retratados continuam altamente citado, e o impacto do autor sobe msm quando ele não publica mais
\item Entretanto, nós somos continuamente avaliados e, mesmo que hajam métricas que são ``infladas'' pelo trabalho cumulativo (citações, por exemplo), há tantas outras q não o são
\end{itemize}
\item Não se deve usar média em bibliometria
\begin{itemize}
\item ..já que a distribuição de citações é extremamente assimiétrica
\item Entretanto, geralmente trabalhamos com amostras (Corpus) consideravelmente grandes de uma totalidade. Daí entra o teorema do limite central.
\end{itemize}
\end{enumerate}
\end{itemize}

\section{Artigo: \cite{garner2018}}
\label{sec:org5fdd91a}
\begin{itemize}
\item Sumarização de várias métricas
\item Excelente explicação do q é o h-index e outros índices
\item Tem tabela excelente que mostra como os índices variam de acordo com a database usada.
\end{itemize}

\section{Artigo: \cite{wallin2005}}
\label{sec:org9aa4ebc}
\begin{itemize}
\item Fala de alguns métodos bibliometricos pouco conhecidos/discutidos: Publication analysis, Bibliographing, etc\ldots{}
\item ``JIF was originally only envisaged as an aid for scientific libraries for the evaluation of their choice of scientific journals.''
\item Citação como medida de qualidade: Implica assumir que TODO MUNDO lê TODA A BIBLIOGRAFIA da sua área e consegue, sem viéses, selecionar apenas os verdadeiramente mais relevantes. Ao mesmo tempo, os viéses se diluem se analisarmos muitas pessoas de uma vez.
\item Cada pesquisador tem sua própria ``identidade de citação'' (White 2001 \& 2004), citando por diferentes motivos
\item ``If a relationship between citation frequency and research quality does exist, this relationship is not likely to be linear. The relationship between research quality and citation fre- quency probably takes the form of a J-shaped curve, with exceedingly bad research cited more frequently than mediocre research (Bornstein 1991)''
\item ``The conclusion must therefore be that there is no unam- biguous relationship between citation parameters and scien- tific importance and/or quality. If we then assume that there must after all be some sort of relationship, an explanation for these clearly conflicting investigations must therefore be that the relationship is so complex that we have difficulty in capturing it with the tools available to us''
\item Matthew effect (Merton 1968) or ``cumulative advantage'' principle (Price 1976): ``In the citation world this effect relates to the fact that citing a publication singles it out for other authors, which increases its chances of being cited again''
\item Foca bastante na normalização e como \textbf{várias comparações podem sim ser feitas, mesmo entre áreas bem distintas, contanto que os dados seja normalizados}
\end{itemize}

\section{Artigo: \cite{belter2015}}
\label{sec:orga504b43}
\begin{itemize}
\item Bela crítica da idéia que citações são medida de impacto. Desconstrução do que ``impacto'' significa, afinal\ldots{}
\item \textbf{A citação pode ser usada para medir o quão útil um artigo foi para outros autores}. Mas meio q é só isso que ele mede\ldots{} E a importância dele para o público em geral? Questões a nível local ou regional?
\item Fala dos problemas do peer-review q levam a usar indicadores: quantidade de papers, falta de reprodutibilidade\ldots{}
\end{itemize}

\section{Artigo: \cite{durieux2010}}
\label{sec:org2616529}
\begin{itemize}
\item Descrição boa de praticamente tudo q dá pra fazer em bibliometria
\item Separação dos indicadores bibliometricos em 3 tipos: Quantidade, qualidade e estruturais.
\item Possivelmente um bom exemplo para a aula:
\begin{itemize}
\item `` For example, J.E. Hirsh has reported that the top 10 researchers in physics and biology have quite different h-indexes (46).''
\end{itemize}
\end{itemize}


\section{Artigo: \cite{mingers2015}}
\label{sec:org70ab11c}
\begin{itemize}
\item Abstract: ``Scientometrics is the study of the quantitative aspects of the process of science as a communication system. \textbf{It is centrally, but not only, concerned with the analysis of citations in the academic literature.} In recent years it has come to play a major role in the measurement and evaluation of research performance. In this review we consider: the historical development of scientometrics, sources of citation data, citation metrics and the “laws'' of scientometrics, normalisation, journal impact factors and other journal metrics, visualising and mapping science, evaluation and policy, and future developments. ``
\begin{itemize}
\item Mostra como não só a database, mas a estratégia de análise altera os índices. (Tabela 1)
\item Entra em alguns aspectos estatísticos (as leis da cientometria)
\item Ponto interessante: se considerarmos à queima roupa que citações são sinonimo de qulaidade, um artigo ter 0 citações significa um artigo sem qualidade e, como boa parte das publicações não são citadas at all, isso significa que teríamos que aceitar que boa parte da ciência produzida é essencialmente lixo.
\item Thomas Khun e como o h-index não faz jus a ele at al, por ele ter poucas publicações
\item h-index usado em diversas outras áreas
\item Toda a literatura concorda que o h-index sozinho é mto cru, e que deve ser usado com outros indicadores.
\end{itemize}
\item Categorias WoS - Aparentemente criticadas:
\begin{itemize}
\item ``This approach has obvious advantages – \textbf{it avoids the use of WoS categories which are ad hoc and outdated (Leydesdorff \& Bornmann, 2014; Mingers, J. \& Leydesdorff, 2014)} and it allows for journals that are interdisciplinary and that would therefore be referenced by journals from a range of fields.''
\end{itemize}
\item A publicação do Journal Impact Factor tem copyright. Não é qqr um q pode calcular e publicar ele.
\item Social Sciences and Humanities - Citation data often not available. In part, because of books being the standard communication vehicle instead of articles. This limits the use of bibliometrics for Evaluation and Policy.
\item Fala de vários drawbacks/vantagens do uso da bibliometria na avaliação e determinação de políticas.
\begin{itemize}
\item ``At this time, full bibliometric evaluation is feasible in science and some areas of social science, but not in the humanities or some areas of technology (Archambault, Vignola-Gagné, Côté, Larivière, \& Gingras, 2006; Nederhof, 2006; van Leeuwen, 2006).''
\item ``Fourth, we must recognise, and try to minimise, the fact that the act of measuring inevitably changes the behaviour of the people being measured. So, citation-based metrics will lead to practices, legitimate and illegitimate, to increase citations; an emphasis on 4* journals leads to a lack of innovation and a reinforcement of the status quo''
\item ``Fifth, we must be aware that often problems are caused not by the data or metrics themselves, but by their inappropriate use either by academics or by administrators (Bornmann \& Leydesdorff, 2014; van Raan, A., 2005b). There is often a desire for “quick and dirty” results and so simple measures such as the h-index or the JIF are used indiscriminately without due attention being paid to their limitations and biases''
\end{itemize}
\item ``One of the interesting characteristics of altmetrics is that it throws light on the impacts of scholarly work on the general public rather than just the academic community.''
\item ``A network can be visualized, but can also be formalized as a matrix'' - Talvez algo interessante de se colocar na apresentação\ldots{}
\item ``There is thus a bifurcation within the discipline of scientometrics. On the one hand, and by far the dominant partner, we have the relatively positivistic, quantitative analysis of citations as they have happened, after the fact so to speak. And on the other, we have the sociological, and often constructivist theorising about the generation of citations – a theory of citing behaviour. Clearly the two sides are, and need to be linked.''
\end{itemize}


\section{Artigo: \cite{mugnaini2014}}
\label{sec:org74f7f63}
\begin{itemize}
\item Qualis: Mesmo em ciências sociais, o artigo costuma ter mais peso que os livros. Há tantas outras áreas (ciência dura, em sua maioria), que não propõem critérios para a classificação de livros.
\item ``Conhecer os critérios de avaliação de um programa de pós-graduação torna-se então dever, tanto dos pesquisadores credenciados, quanto dos alunos, desde seu ingresso, já que disso depende o desenvolvimento do programa.''
\item Seria uma boa idéia apresentar os criterios de avaliação da pós do IBQM?
\item Seria o Qualis e os comitês um meio termo entre analise bibliometrica pura e um tipo (meio tosco) de peer review?
\item ``A avaliação da produção brasileira não se baseia nas citações que sua produção recebe, mas sim nas citações recebidas pelos periódicos onde os brasileiros publicam, principalmente o Fator de Impacto JCR [3a], mesmo considerando literatura extensa sobre suas limitações (ARCHAMBAULT e LARIVIÈRE, 2009; VANCLAY, 2011). Assim, a pouca inserção da produção científica nacional (LETA, 2011) acarreta numa avaliação baseada em indicadores de produtividade, que resulta em produtivismo exagerado, impondo a necessidade de estabelecimento de critérios de qualidade.''
\item ``Como pode-se perceber todas as áreas de avaliação de Biológicas e Engenharias executam a classificação dos periódicos de sua área simplesmente manejando a lista de periódicos e respectivo indicador, tendo que atualizar a lista e os parâmetros de cada estrato, a cada triênio.''
\item Afinal, o quão importante é o JIF no Qualis?
\end{itemize}


\section{Ler artigos mais focados na história da bibliometria/cientometria da próxima vez}
\label{sec:orgde9d6ca}

\section{Artigo: \cite{thompson2015}}
\label{sec:orga62eff3}
\begin{itemize}
\item Foca no histórico e no uso de bibliometria nas ciêcias médicas
\item Tem uma explicação menos matemática das Leis de Lotka e Bradford.
\item ``Of course, all metrics must be used in context. Bibliometric indexes should generally be used in concert with a thoughtful review by senior colleagues.33, 34'' \textbf{OLHAR ESSAS REFERÊNCIAS DPS}
\end{itemize}

\section{Artigo: \cite{araujo2006}}
\label{sec:org8bdcf75}
\begin{itemize}
\item Tbm fala das leis. No caso das 3: Lotka, Bradford e Zipf.
\item Lotka
\begin{itemize}
\item Fala como a lei de Lotka foi criticada, e como houveram improvements a ela, feitos por exemplo por Price.
\item Fala de uma outra lei, a \textbf{lei do elitismo de Price}:
\begin{itemize}
\item `` Logo depois foi formu- lada a lei do elitismo de Price: o número de membros da elite corresponde à raiz quadrada do número total de autores, e a metade do total da produção é considerado o critério para se saber se a elite é produtiva ou não.''
\end{itemize}
\end{itemize}
\item Bradford
\begin{itemize}
\item Lei da dispersão explica pq os índices têm dificuldade em atingir cobertura completa de assuntos. As 2 zonas externas (e especialmente a terceira) possuem um número muito grande de periódicos. Por isso que thompson2015 diz que essa lei inspirou Garfield a criar o SCI, focando em periódicos mais relevantes (core).
\item ``Bradford constatou que mais da metade do total de artigos úteis não estavam sendo cobertos pelos serviços de indexação e resumos''
\item Essa lei tbm foi contestada e reformulada várias vezes.
\begin{itemize}
\item ``Essa lei tam- bém foi sendo constantemente reformulada e aperfeiçoada, como por exem- plo por Vickery, em 1948, que propôs que o número de zonas não precisa ser três mas qualquer número.''
\end{itemize}
\item Lei mto importante para bibliotecas:
\begin{itemize}
\item ``Essa lei foi muito utilizada para aplicações práticas em bibliotecas, tais como o estudo do uso de coleções para auxiliar na decisão quanto à aquisição, descartes, encadernação, depósito, utilização de verba, planejamento de siste- ma.''
\end{itemize}
\item ``Estudos atuais têm sido realizados (COOPER; BLAIR; PAO, 1993) bus- cando identificar core lists, isto é, núcleos de periódicos mais produtivos, de uma determinada área, em revisões que confirmam ou reformulam a Lei de Bradford.''
\end{itemize}
\item Zipf
\begin{itemize}
\item Traça uma correlação entre a ordenação das palavras mais usadas em um texto e sua frequência
\item Um pequeno numero de palavras é usado mais frequentemente. A maioria é bem mais esporádica
\item A posição da palavra multiplicada pela sua frequencia cai em uma constante
\item Zipf forma o princípio do menor esforço:
\begin{itemize}
\item Há economia no numero de palavras usadas. Não há dispersão, mas sim concentração de palavras.
\item Aquelas nuvems de palavras com os termos mais usados representam bem isso.
\item \textbf{As palavras mais usadas indicam o assunto do documento.}
\item Com cada vez mais acesso a full-text documents, isso me parece bem interessante.
\end{itemize}
\item Essa lei tbm foi extremamente revisitada e aprimorada ao longo do tempo.
\end{itemize}

\item ``Uma variação de enfoques bibliométricos é a teoria epidêmica da trans- missão de idéias, desenvolvida por Goffman e Newill, em 1967, que explica a propagação de idéias dentro de uma determinada comunidade como um fe- nômeno similar à transmissão das doenças infecciosas (ou seja, pelo processo epidêmico). Os autores realizam seu estudo por comparação do ciclo da esquistossomose e da informação, fazendo uma analogia entre os dois siste- mas.'' - \textbf{Linka bem com a idéia de ligar bibliometria e epidemiologia de thompson2015.}

\item Daí mergulha na área que considera a mais importante da bibliometria: a análise de citações.
\begin{itemize}
\item Cita 4 tipos de citações distintas.
\item ``Dentro da bibliometria, particularmente a análise de citações permite a identificação e descrição de uma série de padrões na produção do conheci- mento científico. Com os dados retirados das citações pode-se descobrir: au- tores mais citados, autores mais produtivos, elite de pesquisa, frente de pes- quisa, fator de impacto dos autores, procedência geográfica e/ou institucional dos autores mais influentes em um determinado campo de pesquisa; tipo de documento mais utilizado, idade média da literatura utilizada, obsolescência da literatura, procedência geográfica e/ou institucional da bibliografia utiliza- da; periódicos mais citados, “core” de periódicos que compõem um campo.''
\item Fala do histórico e origens da analise de citações e culmina com o SCI de Garfield.
\item Fala tbm do envelhecimento (obsolescência) da literatura
\begin{itemize}
\item Há dois tipos de envelhecimento (clássico ou efêmero)
\item ALgumas áreas tem o envelhecimento clássico presente, enquanto em outras o envelhecimento é bem efêmero (se baseia em literatura recente que em breve será outdated). Outras são intermediárias.
\end{itemize}
\end{itemize}

\item Fala tbm sobre a história da bibliometria no Brasil.
\item E sobre diferenças conceituais entre bibliometria, scientometria e informetria
\item E como a bibliometria foi se transformando cada vez mais em uma técnica, que deve ser adotada em conjunto a outros referenciais e métodos (muitas vezes advindos das ciências sociais).
\begin{itemize}
\item ``São trabalhos que se utilizam de dados bibliométricos mas que realizam uma leitura desses dados à luz de ele- mentos do contexto sócio-histórico em que a atividade científica é produzida.''
\end{itemize}

\item Exemplo interessante de uso da bibliometria fora do âmbito de indicadores de produção científica e afins:
\begin{itemize}
\item ``Diversas frentes de estudo são levadas a termo na atualidade com essa proposta. Há, por exemplo, estudos de usuários feitos com o auxílio de técni- cas bibliométricas. É o caso do estudo de Oliveira (2004), que analisa a possi- bilidade de aquisição de itens para uma biblioteca universitária a partir de indicativos de necessidades de usuários obtidos com o estudo bibliométrico das referências bibliográficas de teses e dissertações.''
\end{itemize}
\item Ele fala de diversas outras aplicações da bibliometria. Bem interessante.
\end{itemize}


\section{Artigo: \cite{urbizagastegui}}
\label{sec:orgf571489}
\begin{itemize}
\item Basicamente, fala como as bases da bibliometria (diversas leis e conceitos, como a obsolescencia, o estudo de ocorrências das palavras e a transmissão de idéias científicas como um modelo epidêmico) já estavam em andamento bem antes da introdução do termo bibliometria em 1969.
\item Começa discorrendo sobre as origens históricas do termo bibliometria.
\item Fala sobre estudos precursores da bibliometria que são bem obscuros e não aparecem na maioria dos outros papers.
\item Lotka, Bradford e esses caras criaram as leis antes mesmo do termo bibliometria ser popularizado em 1969 por Pritchard (lembrando que o primeiro uso foi de otlet em 1934 - \emph{bilbiometrie})
\item ``Em razão do estilo especial e particular de cada falante ou escritor, assim como da existência de uma multiplicidade de línguas, nunca se pensou que a freqüência de ocorrência de palavras num texto tivesse um tipo especial de comportamento. Não obstante, Estoup (1908) já tinha observado que as frequências das palavras da linguagem natural seguem leis estatísticas, tanto que, quando as frequências das palavras são traçadas sobre um papel gráfico, em ordem descendente de freqüências, forma-se uma hipérbole muito similar àquela chamada hoje “Lei de Zipf”.''
\item Daí, fala de indicadores de desenvolvimento da área, como a ISSI e seus congressos internacionais, assim comoo surgimento de periódicos específicos, como o scientometrics em 1978. Após mostrar essas evidências, conclui:
\begin{itemize}
\item `` Enfim, pode-se constatar que a institucionalização e legitimação da Bibliometria está em plena expansão. ''
\end{itemize}
\item Price e a distribuição da vantagem acumulativa. Olhar isso melhor. (Sucesso gera sucesso)
\item Na primeira revisão específica do estado-da-art da bibliometria (Narin \& Moll, 1977):
\begin{itemize}
\item ``Os autores concluíram que os dados bibliométricos proporcionam observações precisas e adequadas sobre o comportamento da informação, sendo seu maior desafio o desenvolvimento de técnicas mais confiáveis e úteis para a avaliação e a predição.''
\end{itemize}
\item Sobre a segunda revisão:
\begin{itemize}
\item ``A segunda revisão, feita por White \& McCain (1989), cobre a literatura produzida de 1977 a 1988. Os autores afirmam que não pretendem \textbf{“explicar de novo as leis de Bradford, Lotka e Zipf, as noções da vantagem cumulativa, acoplamento bibliográfico e co-citação, e assim em diante, mas focalizar as linhas de pesquisas [bibliométricas] emergentes dentro das grandes especialidades”} (White e McCain, 1989: 120). Concluem a revisão afirmando que as possibilidades da Bibliometria merecem maiores oportunidades de exploração, apesar de suas fragilidades''
\end{itemize}
\item Tbm fala por alto do \textbf{modelo de crescimento dos recursos limitados de Shaw}

\item E depois disso finaliza falando sobre as mais diversas divisões da Bibliometria propostas pelos mais diversos autores.
\end{itemize}

\section{Selecionando artigos que falem mais do Brasil (scielo, serrapilheira, qualis)}
\label{sec:org7330c65}
\begin{itemize}
\item Encontrei alguns arigos sobre o scielo, mas tem MUITO mais artigos sobre qualis.
\item Não encontrei nada da serrapilheira.
\item Aproveitei para adicionar alguns papers que falam da relação entre indicadores bibliometricos e peer-review\ldots{}
\item Semana que vem, quando for ler, ordenar por ``date added'' no Zotero
\end{itemize}

\section{Artigo: \cite{rego2014}}
\label{sec:orgdc17e99}

\section{Palestra: Como superar a ciência impaciente e a quantofrenia acadêmica?  Uma proposta para avaliação científica mais humana e responsável - Marcus Oliveira}
\label{sec:orgdcd68c4}
\begin{itemize}
\item Essa questão pode não fazer parte da minha área, mas me afeta (e acredito que afeta todos aqui) diretamente.
\item Práticas comuns do meio corporativo foram adotados/incorporados à nossa cultura de fazer ciencia (além da burocracia): Excelencia, produtividade, impacto\ldots{}
\item Produtividade científica acaba sendo encarada como um fim em si msma
\item Ciência impaciente: Cientistas como CEOs após a incorporação desse jargão corporativista
\item Amplo foco em indicadores numericos para avaliar pesquisadores, instituções e (erroneamente) a qualidade da pesquisa
\item Impatient science - Desejo de retorno rápido de papers, citações, prestígio, etc\ldots{}
\begin{itemize}
\item Aumento da taxa de publicação dos artigos científicos
\item Cientistas ficam sobrecarregados de informações. Nossa leitura fica cada vez mais superficial graças a isso.
\item Além disso, os cientistas tbm estão sobrecarregados de atividades não cientificas. Sobrecarga na rotina de trabalho.
\end{itemize}
\item Quais são as consequencias da ciencia impaciente
\begin{itemize}
\item Burnout (de Meis et al., 2003). Ficamos acabados psicologicamente.
\item Aumento das retratações, principalmente em high-profile journals - Talvez um indicio da pressão para que os pesquisadores publiquem nessas revistas de maior impacto
\item Crise de reprodutibilidade - Muita da pesquisa que vem sendo feita acaba não sendo reprodutivel, já que a pressão é que seja publicável, não reprodutível
\end{itemize}
\item Ciencia impaciente: conhecimento como uma profitable commodity
\begin{itemize}
\item Similar ao capital impaciente
\item O mercado de publicação cientifica se tornou extremamente rentavel/lucrativo
\item Perfeito modelo de business/negócios: Consumidor e produtor é a msma pessoa: nós. E nós não fazemos idéia do quanto de dinheiro circula nesse mercado.
\end{itemize}
\item Challenge \#1: Overcome the current of scientific publishing (the open access push)
\begin{itemize}
\item Low costs, high revenue  boost journal profits
\begin{itemize}
\item Scientific publishing models:
\begin{itemize}
\item Subscription: Universidade paga mensalidade para revistas para ter acesso a seu conteudo
\item Paywall: Pessoa individual paga para acessar artigo/conteúdo, caso a universidade não assine aquela revista
\item Open acces: Uma vez publicado o artigo, qqr um pode ter acesso ao conteudo dele permanentemente. O problema é que o custo de publicação desses artigos é bem alto. E esse custo é mtas vezes pago pelo autor (Author page charges - APC)
\begin{itemize}
\item Modelo verde: Manuscritos disponíveis em repositorios onlines após período de embargo (self-archiving - vc mantém o artigo e pode disponibilizá-lo após um certo tempo)
\item Modelo gold: Manuscritos aceitos são acessíveis a todos após publicação. Extremamente abrangente e lucrativo.
\begin{itemize}
\item Baixo custo:
\begin{itemize}
\item Pesquisadores não são pagos pelas empresas editoriais para produzir conhecimento
\item Pesquisadores escrevem de graça para comunicar seus achados
\item Boa parte da pesquisa publicada é financiada pelo setor público (empresas não arcam com um tostão disso)
\item ``Peer-review'' é feito de graça por outros cientistas.
\item Era digital diminuiu os gastos com publicação
\end{itemize}
\item Alta receita:
\begin{itemize}
\item Taxa de assinatura para universidades são muito altas, assim como paywal (revistas não open-access)
\item Page charge muito alto (APC) nos journals open access (OA)
\end{itemize}
\end{itemize}
\end{itemize}
\item O open access não é tão bonzinho quanto pensamos
\end{itemize}
\item A bizarra industria da publicação científica
\begin{itemize}
\item ``Professional publishers (Elsevier?) add little value to the research process''
\item É tipo pagar impostos altíssimos e o governo terceirizar tudo para lucrar com uma receita insana.
\item Logo, é uma industria bizarra : o Estado financia a pesquisa, paga os salários de quem checa a qualidade da pesquisa (peer-review) e depois ainda paga pela maior parte do produto publicado. Esse modelo foi chamado de ``triple-pay''. Paga três vezes pelo conhecimento, o que não é justo
\end{itemize}
\item Quais são os custos de publicação OA?
\begin{itemize}
\item Nature é o pior caso: 5200 dolares por artigo
\item Em média, o APC (author page charges) é 1800 dólares por artigo
\item Enquanto isso, os editais de financiamento da faperj são de 60000 reais por 3 anos. se publicar 2 papers, 50\% desse dinheiro já vai por agua abaixo.
\item Principais revistas: custo por artigo no modelo gold OA:
\begin{itemize}
\item Nature: 11500 dolares - e em breve só terá o modelo golden open access, por várias razões
\item Cell 10395 dolares
\end{itemize}
\end{itemize}
\item Mercado cientifico é enorme: 2.5 bilhoes de papers baixados por ano (2015) - 2.5x mais q o total de downloads da appstore. O lobo mau sabe disso
\item Open access é um modelo que cai bem às revistas, pois seus artigos têm mais visibilidade e ela lucra o msm (ou até mais) em cima do pesquisador. Para além disso, é garantido que ela vai conseguir lucro com aquele paper específico.
\item A receita total em 2015 da publicação cientifica foi de 10 bilhoes de dolares.
\item OA: 28\% das publicações. Ainda é incipiente.
\item Colizão S (Plan S) - Coalizão feita por editoras em 2018 que visa mudar o modelo de subscription-based para open access. Transferindo o custo dos institutos para o pesquisador. O plano era que todas as pub. cientificas financiadas por isntituiçoes publicas ou privadas deveriam ser OA por volta de 2021. Entretanto, o preprint não entraria nisso.
\item Quem financia o plan S? Várias agências. WHO e outras.Pessoas financiadas por essas agências só podem publicar como OA. O impacto disso é que o pesquisador deve agora arcar com os custos de publicação, usando parte do dinheiro do financiamento para isso, que poderia ir para reagentes, equipamento e afins\ldots{}
\item O plan S está ganhando muita popularidade\ldots{}
\item O plano possui 10 ``mandamentos'':
\begin{itemize}
\item Dentre eles, que os financiadores não devem aceitar o modelo híbrido: publica e não fica aberto durante um tempo de embargo. Depois fica disponível para todos. Mas aceita que revistas q estão em transição receba essas publicações (transformativa arrangements). O que isso quer dizer na prática é que as revistas têm até 2024 para se converter em Full OA. É uma forma de ditar a tendencia do mercado editorial. Se vc tiver um grant de uma agencia financiadora que está no plan S , vc é OBRIGADO a publicar em OA journals (Gold/Green). Vc vai ter que pagar por isso.
\item Tbm fala que as agencias financiadoras devem monitorar se os pesquisadores estão publicando em OA. Senão, irão sofrer sançoes. Tem que tomar cuidado sobre aonde publicar.
\end{itemize}
\item Arranjo transformativo: Forma de corroer o sistema de publicação cientifica - Se uma revista adere ao plan S, ela tem que aumentar o conteudo OA até 75\% 2024. Até 31 de dezembro, elas tem que ser full gold open access.
\item O valor obtido pelo OA será abatido da subscription (como se isso já não fosse uma obrigação) - Na verdade o custo vai aumentar para todo o sistema, como o Marcus fala depois. O menor gasto com periodico capes não necessariamente vai compensar o nosso gasto.
\item Paises low-income tem isenção de APC (waivers), não precisando pagar nada. lower middle-income ganham descontos
\begin{itemize}
\item Brasil é considerado upper-middle economy pelo banco mundial. Não tem desconto. Não teremos isso nesse sistema. A publicação gratuita parece morta. O que se desenha lá fora é mto diferente do que tem aqui.
\item A lista de revistas no plan S é enorme.
\end{itemize}
\item Is Plan S affordable for Brazilian Science?
\begin{itemize}
\item Ok, talvez fosse trocar 6 por meia duzia.
\item Mas aparentemente, o total gasto pelo periodico capes com artigos é menor que o quanto que nós pagaríamos com os page charges.
\item Se o periodico CAPES cobrisse só journal subscription, sairia mais barato, mas ele cobre várias outras coisas.
\item Logo o custo para manter os APCs em conjunto com os periodicos capes vai aumentar.
\end{itemize}
\item Devemos lutar para obter waivers e incluir o Brasil nesse tipo de isenção. Plan S parece uma forma de separar os pesquisadores ricos dos pobres, o norte do sul.
\item Lutar por essa política do waiver é o caminho. Há varias maneiras de lutar contra isso:
\begin{itemize}
\item Demand change on Plan S waiver policy
\item Publish in non-OA (free) journals, independentemente da visibilidade
\item Estimular a publicação em journals locais
\item Publicar em preprints
\begin{itemize}
\item Ok, não é peer-review\ldots{}
\item Será que não se pode pensar em um modelo no qual se publica primeiro e as críticas vêm depois?
\item Marcus acredita que o peer-review funciona, eu tbm (não 100\%, claro)
\end{itemize}
\end{itemize}
\end{itemize}
\end{itemize}
\item Challenge \#2: Overcome academic quantophrenia and the impact factor obsession
\begin{itemize}
\item quantofrenia academica: adesão e obsessão por numeros para explicar qqr coisa. Usar eles para tentar qualificar qqr coisa. - conceito de Pitirim Sorokin, 1965
\begin{itemize}
\item O problema disso é que o fator de impacto é um fim, não um meio. Essa busca incessante pelo fator de impacto, no fim do dia, dá uma nota maior ou menor para o programa de pós graduação. Mas certamente ele está longe de ser uma boa métrica.
\item Fala do Garfield, e de como ele msm diz que o JIF é indicativo, não um valor absoluto. Até que ponto a maior parte da pesquisa não ser tão citada significa que ela não é de qualidade? Mais do que isso, então 99\% da ciência não é de qualidade então? Que trabalho horrível estamos fazendo aqui?
\item O problema não é o fator de impacto (ou qqr outra métrica). O problema somos nós mesmos. Nossa cultura e nosso desconhecimento sobre as métricas e como aplicá-las. Tanto no nível do pesquisador quanto no de instituições e governamental até. Precisa haver uma mudança na nossa cultura.
\item Devemos avaliar o conteúdo do trabalho dos nossos colegas, não o fator de impacto. Isso é difícil, mas é necessário.
\item Fator de impacto usa a média, em vez da mediana\ldots{} Isso, para uma distribuição assimétrica como a das citações científicas, é misleading.
\item Fator de impacto têm seu cálculo muito nebuloso.
\item Há DIVERSOS índices. Mas ele deve ser visto como uma medida acessória, dada a quantidade de publicações. E nenhum índice vai realmente quantificar o prestígio ou qualidade de um pesquisador, da mesma forma que um CR não fala qual aluno é ``melhor''.
\item Deve haver uma balança entre quantitativo e qualitativ, depois jogar a questão: numero de citações indicam qualidade? E perguntar se a maior parte da pesquisa cientifica não é de qualidade então. Se for, o que estamos fazendo aqui?
\item Balancear a avaliação qualitativa e quantitativa. A ciência é mto subjetiva, especialmente na interpretação que damos aos mais diversos conjuntos de dados. As observações podem ser objetivas, mas as extrapolações não o são. A forma como cada um vai contextualizar suas observações é baseada nas experiências e conhecimentos individuais.
\item Peer-review deve ser menos objetiva e mais subjetiva\ldots{} Peer review poderia ser inclusive quantificavel.
\item Como superar a obsessão pelo JIF?
\begin{itemize}
\item Há sugestões na literatura. Ex: \cite{stern2019} - Incluir peer-review como parte do processo de avaliação, para que outros possam avaliar a qualidade da avaliação. Além disso, o desenvolvimento de métricas a nível de artigo podem ser interessantes (incluindo altmetrics)
\item Atribuir um DOI aos reviews, que poderiam então ser citados. Obviamente, o  carater anonimo permaneceria. O fato do peer-review poder receber citação pode inclusive ser um incentivo para que os revisores gastem mais tempo fazendo uma boa revisão. Se os reviews seriam consideravelmente citados já é outra história que não tem como saber\ldots{}
\item Regra dos cinco: Avaliação de pesquisadores. O pesquisador apresenta seus 5 melhores papers dos ultimos 5 anos.
\item O próprio peer review poderia ter algum peso ao se avaliar a produtividade de um pesquisador, para q o mesmo tenha algum incentivo a fazê-lo.
\end{itemize}
\end{itemize}
\end{itemize}
\item Challenge \#3 Rethink where we aspire to publish
\begin{itemize}
\item Journal branding as a market strategy to promote their names to the scientific community
\begin{itemize}
\item As revistas cientificas se tornaram marcas, como Apple, Google, etc.
\item Os grupos editoriais fizeram isso a principio pensando nas redes sociais
\item Em vez de www.scientificreports, o link vai para www.nature, o que estabelece essa associação, esse branding. Daí a gente pensa que só a nature é que deve ser almejada para publicação. Esse marketing infla o prestígio de alguns papers. A gente usa essa maquina para amplificar esse prestígio já exacrebado.
\item Uma possível saída é a publicação em preprints. O inicial foi o arXiv, da área de física. Esse ramo cresceu mto.
\item Mas, infelizmente, nossa comunidade e os tomadores de decisão não abraçaram os preprints, o q é uma pena\ldots{} Preprint é efetivamente um open access real.
\end{itemize}
\end{itemize}
\item Pioneer initiatives to improve scientific assessment
\begin{itemize}
\item Visam melhorar nossa cultura de avaliação
\begin{itemize}
\item San Francisco Declaration on Research Assessment (DORA) - Não usar metricas de jornais para proxy de qualidade de artigos individuais.
\item Leiden Manifesto - Compilado de princípios. Visa a melhoria dos critérios de avaliação.
\item The Metric Tide - Redesenho dessas iniciativas, voltadas à realidade inglesa. Recomenda reduzir o peso das métricas na avaliação das pesquisas.
\item Hong Kong Principles  \cite{moher2020} - Artigo discutido pelo Olavo no reproducibilitea de abril. Baixei a live para ver depois. Reformulação desses critérios, que visa avaliar de maneira correta os pesquisadores. Tornar o processo mais humanitário.
\item IRSA - The Iniciative for Responsible Scientific Assessment - Iniciativa criada pelo Marcus e colegas.
\begin{itemize}
\item Escreveram o seu próprio manifesto.
\item Primeiro ponto é ler os manifestos anteriores e pensar neles. Pensar mais na qualidade de um trabalho do q aonde ele será publicado\ldots{} Esquecer um pouco os índices.
\begin{itemize}
\item A UFRJ e a FAPERJ, o CNPQ, nenhum deles endossou o DORA. Isso não faz sentido.
\end{itemize}
\item Segundo ponto: Aumentar a qualidade do peer review.
\begin{itemize}
\item Tanto fazer boas revisões como sugerir bons revisores.
\item \textbf{Ser subjetivo não significa não ser transparente.}
\item Próprio Garfield fala de como no mundo ideal nós deveríamos ler cada artigo e fazer os julgamentos.
\item Mais uma vez, com a internet, nós conseguiriamos ter acesso às críticas feitas ao artigo, podendo nos basear menos em citações.
\item \textbf{É impossível colocar toda a dimensão de um artigo, de conhecimento gerado, em um único número.}
\end{itemize}
\item Terceiro ponto: Recognize seminal findings by researches
\begin{itemize}
\item Cite primery literature, avoiding reviews.
\item Value central scientific discoveries in funding and hiring decisions.
\item Dá crédito ao pesquisador q trabalhou orginalmente
\end{itemize}
\item Quarto ponto: Promote actions to associate quality assessment with more representative scientometric tools in academic decisions.
\begin{itemize}
\item O fator de impacto não é a melhor métrica.
\end{itemize}
\item Quinto: Submit manuscripts to journals with editors who are active scientists, backed by scientific societies.
\begin{itemize}
\item Particularmente interessante hj em dia, que muitas revistas tem editores profissionais, não cientistas praticantes.
\item Não dar suporte a revistas predatórias.
\item Use pre-prints.
\end{itemize}
\end{itemize}
\item IRSA milestones - DORA endorsou o IRSA. Petição online. O que nós vamos fazer a partir de 2024?
\end{itemize}
\end{itemize}
\item Perguntas:
\begin{itemize}
\item Forma de barganha: Se vc é revisor, ganha capital para conseguir um weaver. Isso torna o parecer do revisor mais impactante. Se adicionar o DOI ao parecer, ajuda nessa barganha. J́á tá tudo ali registrado.
\item Plan S: tira as barreiras dos leitores, mas adiciona barreiras para os autores. Isso desfavorece países como o Brasil. Nós seremos isolados.
\item Questões de integridade científica estão aflorando tbm. O problema do Qualis, que usa fator de impacto primariamente. Se não nos debruçarmos sobre o Plan S, o Qualis vai acabar. Pq basicamente não vamos ter dinheiro para publicar nessas revistas. Sonia fala que o IBqM é mto crítico e tenta contribuir com os APCs.
\item Como cultivar a cultura em um instituto onde tem as amarras? Mudar a cultura é difícil. Como fazer isso?
\begin{itemize}
\item Não precisa recriar nada, mas se a gente tentar seguir boa parte do que o DORA fala, por exemplo. FAPERJ não assinou, por exemplo.
\item Antes de mais nada, devemos ter noção que há problemas na avaliação da ciência como ela é feita atualmente.
\item E que aceitar isso de cabeça baixa implica aceitar um status quo que nos isola do circulo de publicação cientifica e favorece os grandes grupos editoriais (35\% da margem de lucro é absurdo). Para além disso, as informações circulam e são amplificadas por um grupo pequeno de pesquisadoes (Lotka).
\item Isso deve ser ventilado continuamente. Não importa a revista em q vc publica. O próprio Marcus diz q tem um paper da Nature de 20 anos atrás e não é o paper mais citado. Mas a cultura é difícil de ser mudada, em grande parte pq a avaliação quer sumarizar tudo em um numero e ngm quer ficar para trás. Para professores que já estão estabelecidos. Para os professores mais jovens, e principalmente para os alunos, isso não é tão simples. Hj em dia nem os preprints são reconhecidos.
\item Pq o Lattes não aceita preprint? Só pq não tem ISSN? Modelo do preprint é esse: publica primeiro, avalia depois. Será que é um modelo tão ruim assim?
\end{itemize}
\item E o caso da ciencia básica é outro: e quando uma pesquisa básica é incorporada em patentes, mas não tem muitas citações?
\item Vai chegar uma hora que a conta não vai fechar. Daí como que vamos publicar artigos, se teremos que pagar.
\item O problema do qualis somos nós. Afinal, tem diversas comitês para determinar o que é usado na avaliação. Claro que podem haver realocações de revistas, mas elas são as exceções, não a regra. A maioria dos comitês se baseiam principalmente no fator de impacto, q é a regra, para determinar se é A1, A2, etc\ldots{} E o posicionamento de revistas que estão entre A1 e A2, por exemplo, às vezes sobe ou desce por mudanças na ultima casa decimal do fator de impacto, q para fins praticos nao quer dizer nada.
\item \textbf{Idéia interessante: Revistas foram espertas. Ao mudar para o Open Access, eles garantem o pagamento de uma vez e evitam problemas com a pirataria, como é o caso do sci-hub.} A revista já tá garantida a priori. Não precisam mais se preocupar com o scihub\ldots{}
\item Além disso, quando não havia muitos artigos abertos, nada impedia vc de ir pedir o paper para o autor, o que implica perda de receita da revista. Logo, é uma jogada de mestre das revistas. Elas provavelmente vão aumentar seus lucros.
\item O problema de verdade é para quem está abaixo do equador economico. Nós, no caso\ldots{}
\item O problema principal é a cultura, na verdade. Temos de fazer nossa parte. O q o Marcus faz:
\begin{itemize}
\item Não dá parecer para revista open access.
\item As revistas dependem de bons estudos para manter o fator de impacto. Logo, ele tbm não submete nessas revistas se não tiver full waiver.
\item Esquecer revistas top tier, que gera quantofrenia e favorece os ganhos astronomicos das editoras.
\end{itemize}
\item O problema de usar o bioarxiv:
\begin{itemize}
\item Vc não vai conseguir financiamento
\item Vc não consegue adicionar no lattes (bizarramente) - Mas será que isso não dá pra resolver relativamente fácil?
\item Mas como vencer a rede de divulgação das revistas tbm?
\item O convencimento dos alunos deve vir do exemplo tbm. Devemos publicar no bioarxiv e discutir papers do bioarxiv. Mais importante que isso, CITAR esses artigos. Para além disso, devemos comentar sobre os papers que lemos. Raramente os papers do bioarxiv tem algum comentario. Mais uma vez, um problema de cultura. Talvez assim, iremos fazer com que os alunos publiquem ali e, com essa ação mais coletiva, eventualmente, a ficha dos orgaos de fomento irá cair.
\end{itemize}
\end{itemize}
\end{itemize}


\section{Exemplos nos quais bibliometria pode ser usada noa auxílio do peer-review}
\label{sec:org3ac3b44}

\subsection{\cite{juznic2010}}
\label{sec:org89b9a2c}
\begin{itemize}
\item Fala do dual system of grant approval da slovenia - usa tanto bibliometria como peer review.
\item ``Bibliometrics could there serve as an indicator of conflict of interest, indicated for at least 16\% of the projects in 2003. On the other hand, scientometric indicators can hardly replace peer review as the ultimate decision-making and support system''
\item ``An important reason for introducing the dual system of grant approval in 2008 was to decrease the burden of administration, at least for the majority of researchers who already have a rich bibliographic record to prove their excellence. At least half of the researchers that are selected for phase two can be pre-selected using bibliometric methods. ''
\end{itemize}


\subsection{\cite{besselaar2020}}
\label{sec:orgee11268}
\begin{itemize}
\item In this paper, we describe an interesting case in which the use of bibliometrics in a panel-based evaluation of a mid- sized university was systematically tried out. The case suggests a useful way in which bibliometric indica- tors can be used to inform and improve peer review and panel-based evaluation.
\item Today’s reality is that peer review and bibliometric assessment are not anymore two separate activities – in prac- tice they have been merged: many peer reviewers and review panel members use bibliometric databases like WoS- Clarivate, Scopus, Dimensions, Microsoft Academic, Google Scholar, or even ResearchGate to obtain an impression of the applicants or research units they need to evaluate (Moed 2005, ch. 18.3; c.f. de Rijcke \& Rushforth 2015).
\item For professional bibliometricians it is obvious that this non-professional use of bibliometrics may lead to serious prob- lems, as the latter often uses indicators like the journal impact factor and the H-index, which are considered flawed by  the former.
\item Não tenta comparar resultados baseados em  índices bibliométricos com peer-review (que é o que geralmente é feito)
\item Como a bibliometria e o peer-review são mto diferentes (bibliometria tem dificuldade em criar indicadores para qualidade e peer-review possui um certo grau de subjetividade), talvez o melhor seja combinar os pontos positivos de cada um, em vez de tentar fazer com que um suplante o outro. Logo, trabalhos como esse me parecem mais interessantes do que a discussão tão presentes do ``qual é melhor?''
\item ``In order to bring peer review to a level of disinterestedness and fairness (Merton 1973), and to avoid many of the problems of subjectivity and bias that research on peer review has reported, it would be a challenge for the bibliometric community to produce a larger set of valid indicators covering the more quality dimensions that are important when evaluating research, including quality indicators for applied research and societal impact. The current dominance of impact and productivity indicators is too narrow. ''
\end{itemize}


\section{IDEIAS}
\label{sec:org84b7cd7}

\begin{itemize}
\item Falar sobre a institucionalização da ciência e como o lab adotou uma ideologia capitalista (desenterrar os textos do Rômulo)
\item Contar a história de quando eu fui publicar o paper do mestrado e, ao pedir auxílio financeiro à pós-graduação, a mesma estabelecia um ponto de corte com base única e exclusivamente no fator de impacto. No fim conseguimos um desconto e tal\ldots{}
\begin{itemize}
\item Mas será que é sensato, especialmente levando em conta que os recursos são extremamente escassos e que uma boa alocação deles para artigos de qualidade é essencial? E um artigo publicado em revista de maior impacto não necessariamente significa maior numero de citações? Posso falar do paper que fala sobre preditores. Talvez ler mais artigos sobre predição de citações.
\item Já com relação a isso, se o fator de impacto da revista não pode ser aplicado para determinar qualidade de papers individuais e, muito mais importante, não pode ser comparado entre áreas distintas (como fica o Peged nesse caso?), será que é realmente justo usar apenas ele?
\item Será que um tipo de peer review sobre a qualidade dos artigos não seria mais interessante? Se a gente faz a avaliação da qualidade de graça para as revistas cientificas, pq não o fazer para a nossa instituição tbm? Pq não priorizar nossa casa, aliás?
\item Se nós não nos preocuparmos com isso, ngm vai.
\end{itemize}
\item Citações são diferentes entre si: algumas são para isso, algumas para aquilo. E se houvesse uma forma de classificação de citações, para atribuir pesos diferentes a elas? Agora, com uma cada vez maior disponibilidade de artigos full-text, isso se torna cada vez mais uma possibilidade.
\item Pq as pessoas continuam usando tanto o IF? Bom, pq as pessoas continuam adotando 0,05 como limiar de significância para teste estatístico? Meio que se tornou uma convenção. Algo que as pessoas nem param para pensar sobre.
\item Será que tem algum artigo que eu possa usar no arquivo de srtigos do reproducibilitea? Sim.
\item Encontrar artigos sobre:
\begin{itemize}
\item Os manifestos sobre a bibliometria
\item Os bancos de dados usados
\item Softwares usados (citnetexplorer, publish or perish\ldots{})
\end{itemize}
\item Tentar pesquisar no web of science/scielo/scientometrics por artigos de revisão
\item Pesquisar depois qual o fator de impacto para que a pós pague pelo artigo. E discutir que usar só isso como critério é um mal uso do fator de impacto, pois publicar numa revista de alto impacto não significa que o artigo é de qualidade ou que será bem citado.
\item Nas métricas bibliometricas para revistas, falar pelo menos do IF e (claro) do QUALIS
\item Mostrar (com exemplos) que muito da vida moderna envolve caracterização de atividades humanas em termos de estatísticas. Daí falar que o mesmo é válido para a ciência.
\item Problema editorial: Quando os pesos são diferentes por revista, e isso é usado para dar peso à citações, isso não torna o IF algo recursivo? Como que uma nova revista irá surgir e aumentar seu impacto sem as falcatruas dos editoriais e afins?
\item Como o fator de impacto é uma consequencia e tbm uma causa de citações, pode ser complicado trabalhar (só) com ela.
\item Seria uma boa idéia apresentar os criterios de avaliação da pós do IBQM?
\item Criar um programinha que jogue um dado multiplas vezes e tire as médias\ldots{} Para exemplificar o Teorema Central do Limite
\item Comparar os fatores de impacto de duas revistas: uma de molecular e outra de sei lá, ciencia da computação
\item Guardando as devidas proporções, esse lance do fator de impacto é tipo a nota de corte do ENEM. Muitas vezes a pessoa escolhe uma revista por causa do fator de impacto, mesmo que a revista não tenha tanto a ver com a sua pesquisa. Obviamente, espera-se que o peer review dê uma barrada nisso, mas ainda assim\ldots{}
\item Falar do Scielo, Qualis e SERRAPILHEIRA
\item \textbf{Idéia artigos escolhidos:} Um sobre a história da bibliometria, um sobre os diferentes indicadores. 1 ou 2 sobre o cenário brasileiro (qualis, serrapilheira, Scielo, etc.), talvez um sobre altmetria
\item O problema disso é que o fator de impacto é um fim, não um meio. Acho justo buscar um maior fator de impacto maior no sentido de buscar maior visibilidade para o seu trabalho, para que assim ele possa ser mais visto e, esperançosamente, mais útil para a comunidade científica. Entretanto, até que ponto nós não olhamos só para o fator de impacto em vez de pensar onde que o nosso paper irá cumprir melhor a função dele de comunicar nossos achados ao \textbf{público alvo}? Sendo que esse publico alvo não é toda a comunidade acadêmica, mas um subset muito restrito do mesmo\ldots{}
\item Fala do Garfield, e de como ele msm diz que o JIF é indicativo, não um valor absoluto. Até que ponto a maior parte da pesquisa não ser tão citada significa que ela não é de qualidade? Mais do que isso, então 99\% da ciência não é de qualidade então? Que trabalho horrível estamos fazendo aqui? Se considerarmos a citação como medida de qualidade, estamos lascados\ldots{}
\item O problema não é o fator de impacto (ou qqr outra métrica). O problema somos nós mesmos. Nossa cultura e nosso desconhecimento sobre as métricas e como aplicá-las. Tanto no nível do pesquisador quanto no de instituições e governamental até.
\item Há DIVERSOS índices. Mas ele deve ser visto como uma medida acessória, dada a quantidade de publicações. Entretanto, o crescimento exacerbado da quantidade de publicações, por sua vez, está intrinsecamente associado à supervalorização de índices sem a devida consiração do que eles realmente significam.
\item Falar do h-index, que supostamente é uma medida que engloba qualidade e quantidade de publicações, depois jogar a questão: numero de citações indicam qualidade? E perguntar se a maior parte da pesquisa cientifica não é de qualidade então. Se for, o que estamos fazendo aqui?
\item Ao falar das soluções, poderia falar sobre como o fato das métricas não serem relativas aos artigos, mas mtas vezes às revistas é um problema, e como que medidas não a nível de pesquisador ou revista, mas a nível de artigos, seriam muito interessantes. Mais sobre article-level metrics:  \href{https://sparcopen.org/our-work/article-level-metrics/\#:\~:text=Article\%2DLevel\%20Metrics\%20(ALMs),proxy\%20for\%20that\%20publication's\%20importance.}{Article Level Metrics - SPARC} ,   \href{https://web.archive.org/web/20140313155739/http://www.sparc.arl.org/sites/default/files/sparc-alm-primer.pdf}{Wayback Machine} - Poderia usar isso para linkar com altmetrics\ldots{} \href{https://en.wikipedia.org/wiki/Article-level\_metrics}{Article-level metrics - Wikipedia}
\item Atribuir um DOI aos reviews como forma de permitir a citação dos mesmos. Já há inúmeros avanços nisso com relação ao Zenodo, que permite atribuição de DOIs a códigos/research artifacts, então pq não abrir isso para peer-reviews então?
\item Lembrar de falar que a bibliometria pode ser usada para fazer n outras coisas, como avaliar quais papers são mais interessantes, traçar a evolução de um tópico por meio dos seus artigos principais, etc\ldots{}
\item Na parte de histórico, falar da universidade de Leiden, para depois encaixar isso com o Manifesto de Leiden.
\end{itemize}

\section{Escolhendo os papers/criando esqueleto da apresentação}
\label{sec:org36010b1}

\begin{itemize}
\item \textbf{Idéia artigos escolhidos:} Um sobre a história da bibliometria, um sobre os diferentes indicadores. 1 ou 2 sobre o cenário brasileiro (qualis, serrapilheira, Scielo, etc.), talvez um sobre altmetria

\item Historia da bibliometria:
\begin{itemize}
\item \cite{araujo2006}
\item \cite{thompson2015}
\end{itemize}
\item Índices bibliométricos:
\begin{itemize}
\item \cite{garner2018}
\item \cite{durieux2010}
\item \cite{roldan-valadez2019} (meu favorito até agora)
\item Um dos q fala de indices bibliometricos tbm fala de diferenças do seu calculo em databases, certo? Olhar isso com calma.
\end{itemize}
\item Os manifestos
\begin{itemize}
\item DORA: \cite{cagan2013}
\item Leiden Manifesto: \cite{hicks2015}
\item Hong Kong Principles: \cite{moher2020}
\item Changing how we evaluate research is hard, but not impossible (based on DORA)  \cite{hatch2020}
\end{itemize}

\item Peer-review vs bibliometria
\begin{itemize}
\item \cite{besselaar2020} (research) - \textbf{MELHOR NÃO}
\item Alternativamente, poderia citar que os dois possuem fraquezas e forças complementares e citar \cite{besselaar2020} e \cite{haeffner-cavaillon2009a}.
\end{itemize}

\item Brasil
\begin{itemize}
\item \cite{mugnaini2019} (research)
\item Algum paper sobre o Qualis?
\item Lattes tbm pode ser usado para estudos bibliométricos\ldots{} (mugnaini2019) - é um ótimo artigo para apresentar como research article se eu conseguir linkar ele com alguma coisa, como o qualis
\item ``É preocupante o fato de que, na avaliação de programas de pós-graduação no Brasil – mais especificamente no Qualis –, as bases de dados bibliográficas comerciais representam quase a totalidade de critérios de avaliação de produção científica, servindo como: (1) fontes exclusivas de indicadores de impacto – principalmente os índices de citação Scopus e WoS –, quando se trata das áreas associadas às ditas ciências duras, crescentemente assolando as humanas e sociais; e (2) parâmetro de qualidade de periódicos, levando em conta seu processo seletivo – aspecto que se observa de maneira generalizada entre as áreas (Mugnaini, 2015). ''
\item Dá para falar do mugnaini em pelo menos dois momentos: quando for falar das zonas de bradford e quando for falar do qualis. Mesmo se eu acabar não usando ele como um dos 5, dá pra citar ainda assim como trabalho recente que usa essas zonas. Lembrar de atentar nesse momento que há diferenças entre a quantidade de papers nas zonas de cada área, e que diferenças entre diferentes áreas serao mto importantes mais pra frente, quando for falar dos indicadores.
\item \cite{jaffe2020} tbm parece um bom artigo para se ler sobre o qualis, e é research.
\end{itemize}
\end{itemize}


\begin{itemize}
\item Altmetria
\begin{itemize}
\item \cite{bornmann2014}
\item \cite{williams2017}
\item \cite{thelwall2013} (research - precisaria estudar a distribuição Z, bonferroni e afins novamente)
\end{itemize}
\end{itemize}


\section{Amanhã: j}
\label{sec:org40657ac}
\begin{itemize}
\item Ler as anotações
\item Montar um esqueleto bem básico da aula
\item Definir o research article. se mugnaini e jaffe não forem interessantes, olhar a revisão e tentar tirar das suas referencias. Definir os outros será mais de boas dps.
\end{itemize}




\bibliography{Bibliometry}
\bibliographystyle{apalike}
\end{document}
